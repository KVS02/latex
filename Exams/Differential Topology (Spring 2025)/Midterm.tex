\documentclass{amsbook}

\usepackage{amsmath, amsthm, amssymb, amsfonts}
\usepackage{thmtools}
\usepackage{graphicx}
\usepackage{setspace}
\usepackage{geometry}
\usepackage{float}
\usepackage{hyperref}
\usepackage[utf8]{inputenc}
\usepackage[english]{babel}
\usepackage{framed}
\usepackage[dvipsnames]{xcolor}
\usepackage{tcolorbox}
\usepackage{mathrsfs}
\makeatletter
\renewcommand*\env@matrix[1][*\c@MaxMatrixCols c]{%
  \hskip -\arraycolsep
  \let\@ifnextchar\new@ifnextchar
  \array{#1}}
\makeatother


\colorlet{LightGray}{White!90!Periwinkle}
\colorlet{LightOrange}{Orange!15}
\colorlet{LightGreen}{Green!15}

\newcommand{\HRule}[1]{\rule{\linewidth}{#1}}

\declaretheoremstyle[name=Theorem,]{thmsty}
\declaretheorem[style=thmsty,numberwithin=section]{theorem}
\tcolorboxenvironment{theorem}{colback=LightGray}

\declaretheoremstyle[name=Lemma,]{lemsty}
\declaretheorem[style=lemsty,numberwithin=section]{lemma}
\tcolorboxenvironment{lemma}{colback=LightGray}

\newenvironment{solution}[1][Solution]{\proof[#1]\renewcommand{\qedsymbol}{$\blacksquare$}}{\endproof}

\theoremstyle{theorem}
\newtheorem{proposition}[theorem]{Proposition}

\theoremstyle{plain}
\newtheorem{definition}[theorem]{Definition}
\newtheorem{example}[theorem]{Example}
\newtheorem{exercise}[theorem]{Exercise}
\newtheorem{notation}[theorem]{Notation}
\newtheorem{question}[theorem]{Question}
\newtheorem{problem}[theorem]{Problem}
\newtheorem{corollary}{Corollary}[theorem]

\theoremstyle{remark}
\newtheorem{remark}[theorem]{Remark}
\newtheorem{remarks}[theorem]{Remarks}

\setstretch{1.2}
\geometry{
    textheight=9in,
    textwidth=5.5in,
    top=1in,
    left=1in,
    right=1in,
    headheight=12pt,
    headsep=25pt,
    footskip=30pt
  }

  \newcommand{\Q}{\mathbb Q}
  \newcommand{\Z}{\mathbb Z}
  \newcommand{\F}{\mathbb F}
  \newcommand{\R}{\mathbb R}
  \newcommand{\PP}{\mathscr P}
  \renewcommand{\P}{\mathbb P}
  \newcommand{\Aut}{\mathrm {Aut}}
  \newcommand{\Gal}{\mathrm {Gal}}
  \newcommand{\GL}{\operatorname{GL}}
  \newcommand{\stab}{\operatorname{Stab}}
  \newcommand{\p}{\mathfrak p}
  \newcommand{\m}{\mathfrak m}
  \renewcommand{\d}{\operatorname d}
  \newcommand{\cbrt}[1]{\sqrt[3]{#1}}





  

% ------------------------------------------------------------------------------

\begin{document}

% ------------------------------------------------------------------------------
% Cover Page and ToC
% ------------------------------------------------------------------------------

\title{ \normalsize \textsc{}
		\\ [2.0cm]
		\HRule{1.5pt} \\
		\LARGE \textbf{\uppercase{Midterm}
		\HRule{2.0pt} \\ [0.6cm] \LARGE{Differential Topology} \vspace*{10\baselineskip}}
		}
\date{}
\author{\textbf{Sudharshan K V} \\ 
		\today}

\maketitle
\chapter*{Problem 1}
Any matrix $A$ for which $A^TDA = D$ must be invertible, as taking determinants shows. Therefore, it suffices to show $\stab(D)$ is a submanifold of $\GL_n(\R)$. We will show this using the preimage theorem. Note that $A^TDA$ is always symmetric.\\

Consider the map $F: A \mapsto A^TDA$ from $\GL_n(\R)$ to the set of symmetric matrices. This map is a polynomial map, hence smooth. We will show that $D$ is a regular value for this map. Indeed, at a point $A$, we may compute the derivative of $F$ as follows.

\begin{align*}
  DF_A(B) &= \frac{\d}{\d t} |_{t=0}\left( F(A+tB) - F(A)\right)\\
          &= \frac{\d}{\d t} |_{t=0}\left( t B^TDA + t A^TDB + t^2 B^TDB \right)\\
          &= B^TDA + A^TDB.    
\end{align*}
One can see that the tangent space of symmetric matrices is the subspace of symmetric matrices in $M_n(\R)$, owing to the submanifold of symmetric matrices itself being a linear subspace of $M_n(\R)$. We will show that the image of $DF_A$ is the whole tangent space when $A^TDA = D$. \\

Indeed, let $M$ be any symmetric matrix. Choose \[B = D^{-1}(A^T)^{-1} \frac M 2.\] One then computes

\begin{align*}
  DF_A(B) &= B^TDA + A^TDB\\
          &= \frac{M^T} 2 + \frac M 2\\
          &= M.
\end{align*}
Therefore the derivative of $F$ is surjective at all points of $F^{-1}(D)$. By the preimage theorem, $\stab(D) = F^{-1}(D)$ is a submanifold of $\GL_n(\R)$ with dimension $n^2 - \frac{n(n+1)}2 = \frac{n(n-1)}2$, since $\frac{n(n+1)}2$ is the dimension of the space of symmetric matrices. \\

One also computes that $DF_A(D^{-1}(A^T)^{-1}C) = C^T + C$, so the kernel of $DF_A$ is the space of skew symmetric matrices multiplied by $A^TD$ (when identified as a subspaces of $\R^{n^2}$). The kernel of the differential is the tangent space of $\stab (D)$. \\

We exhibit two matrices $D_1, D_2 $ of the same size for which $\stab D_1$ and $\stab D_2$ are not diffeomorphic. Let \[D_1 =
  \begin{pmatrix}
    1& 0\\ 0& 1
  \end{pmatrix}, D_2 =
  \begin{pmatrix}
    1& 0 \\ 0 & -1
  \end{pmatrix}.\]
Then $\stab(D_1) = \{A\in M_2(\R): A^TA = I\}$ is compact, since it is closed and bounded. The boundedness comes from computing $A^TA$, which forces the columns to have norm $1$, bounding all entries of the matrix between $\pm 1$.\\

Let $c,d$ be large positive real numbers such that $d^2 - c^2 = 1$ (note that such choices can be arbitrarily large). Let $a,b$ be the positive real roots of $c^2+1$ and $d^2 - 1$. Then one has the relation \[(c^2 + 1)(d^2 - 1) = c^2d^2,\] so we have
\begin{itemize}
\item $a^2b^2 = c^2d^2$, hence $ab = cd$.
\item $a^2-c^2 = 1$.
\item $b^2 - d^2 = -1$.
\end{itemize}
When $A =
\begin{pmatrix}
  a&b\\c&d
\end{pmatrix}$, one computes using the three listed relations that $A^TD_2A = D_2$. Therefore, $\stab (D_2)$ is an unbounded closed subset of $\R^{n^2}$. In particular, the stabilizer is not compact and is therefore not diffeomorphic to $\stab(D_1)$.
\chapter*{Problem 2}
\section*{(a)}
We will provide a counter example to show that $p$ need not be a regular value of $f$. Let $X$ be the $2$-sphere, \[X = \{(x,y,z) \in \R^3 : x^2 + y^2 + z^2 = 1 \}.\] Let $\pi: \R^3 \to \R$ be the square of the $z$-coordinate map, i.e., $\pi(x,y,z) = z^2$. Let $Y = \R$ (so $n = 1$), and let $f$ be the restriction of $\pi$ to $X$. So $f = \pi i$, where $i:X\to \R^3$ is the inclusion. Let $p = 0$. In this case, $f^{-1}(p)$ is the circle in $X$ along the $xy$ plane, and is therefore a submanifold of positive dimension with codimension $n$. 

Observe that \[D\pi_{x,y,z} =
  \begin{pmatrix}
    0 \\ 0 \\ 2z
  \end{pmatrix}\] in standard coordinates. Therefore, at any point $x\in X$ with $z$-coordinate $0$, one has that $Df_x = Di_x D\pi_x = 0$. In particular, this means that $p = 0$ is not a regular value of $f$.
\section*{(b)}
Since $X$ is compact, one has $f(X)$ bounded. Therefore, if $v$ is a vector in some orthogonal direction of $W$, then for some large real number $c$, $W + cv$ does not meet $f(X)$. In this case, $f^{-1}(W+cv)$ is the empty set which is a submanifold of $X$.

\section*{(c)}
Let $f:X \to \R^n - 0$ miss the origin. Let $\pi: \R^n - 0 \to \R\P^n$ be the natural map, given by \[\pi(x_1, \dots, x_n) = [x_1:\dots:x_n].\] We will show that $\ell$ is transverse to $f$ if and only if $\pi (\ell)$ is a regular value of $\pi \circ f$. This is similar to how one shows that a ray $r$ is transverse to $f$ iff the direction of $r$ is a regular value in $S^{n-1}$ (in the proof of Jordan Brower Theorem, HW7).

\begin{lemma}
  Let $v = (v_1,\dots, v_n)$ be a vector in $T_p(\R^n - 0)$, where $p = (p_1, \dots, p_n)$. Then, \[D\pi_p(v) = 0 \text{ if and only if there is some } c \in \R, v = cp.\]
\end{lemma}
\begin{proof}
  Assume $p_1 \neq 0$ without losing generality. We do this in the standard chart around $\pi(p)$. The chart around $[p_1:\dots:p_n]$ is $[x_1:\dots:x_n] \mapsto (x_2/x_1, \dots, x_n/x_1)$. In this chart, we compute the derivative map to be \[D\pi_p =
    \frac{1}{p_1}\begin{pmatrix}[c|ccc]
                   -\tfrac{p_2}{p_1} & &&\\
                   \vdots &  & I &\\
                   -\tfrac{p_n}{p_1}&&&
    \end{pmatrix},\] where $I$ is the size $n-1$ identity matrix. Then evaluating at $v$ and setting $c = v_1/x_1$ (which is well-defined), we find that $D\pi_p(v) = 0$ if and only if $v = cp$.
\end{proof}
Consequently, observe that $\pi$ is a submersion and the kernel of $D\pi_p$ is just the line $\ell$ passing through $0$ and $p$. 
\begin{lemma}
  Let $\ell$ be a line through the origin in $\R^n$. Then, $\pi(\ell)$ is a regular value of $\pi\circ f$ if and only if $f$ is transverse to $\ell$. 
\end{lemma}
\begin{proof}
  Let $x$ be a point in $x$ such that $f(x) \in \ell$. Transversality of $\ell$ and $f$ is equivalent to $Df_x(T_xX) + \ell = \R^n$ for all such $x$. Let $T$ be the tangent space of $\R\P^n$ at $\pi(f(x))$. The condition $Df_x(T_xX) + \ell = \R^n$ implies $D(\pi\circ f)_x(T_xX) = T$, since $\pi$ is a submersion, and $\ell$ is the kernel of $D\pi_{f(x)}$. Conversely, if $D(\pi\circ f)_x(T_xX) = T$ then $Df_x(T_xX) + \ell = \R^n$ for the exact same reason. Furthermore, $f(x) \in \ell$ if and only if $\pi(f(x)) = \pi(\ell)$. So transversality at all such $x$ is equivalent to the regularity of $f(\ell)$. 
\end{proof}

Now we may choose a regular value of $\R\P^{n-1}$ due to Sard's theorem (even when $n = 1$!). This point yields a line in $\R^n$ which is transverse to $f$.

\chapter*{Problem 3}

All spaces we deal with will be locally Euclidean, hence locally path connected. So we will not distinguish path connectedness from connectedness. Let $\dim Y = n$, $\dim X = n-1$. 

\section*{(a)}

We repeat the Jordan Brouwer argument, but modify the proof for non-emptiness of a certain set.\\

Fix a point $z \in Y$ which is not in $X$. Consider the property $\PP$ on $X$ given by \[\PP: \text{ For all neighborhoods $U$ of $x$, there is a path connecting $z$ to some point in $U$ which does not meet $X$}.\] Let $x$ be a point of $X$ which satisfies $\PP$. Choose a chart $U$ around $X$ so that $X\subset Y$ in this chart is locally is $\R^{n-1} \subset \R^n$. Then $U-X$ has two (path) connected components, say $U_1$ and $U_2$. By $\PP$, there is some point in $U-X$ which is path connected to $z$ in $\R^n-X$. Say this point is in component $U_1$. Then all points of $U_1$ are path connected to $z$ in $\R^n-X$. For any neighborhood $V$ around any point in $U\cap X$, there is a point in $U_1 \cap V$, so every point of $U\cap X$ possesses property $\PP$. It follows that the subset of points of $X$ which satisfy $\PP$ is open.\\

Now if $x$ is a point which does not satisfy $\PP$, then there must be some open neighborhood $U$ of $x$ for which no point of $U$ is path connected to $z$ in $\R^n-X$. Taking $U$ itself as the neighborhood, we see that each point of $U\cap X$ does not satisfy property $\PP$. Therefore, the set of points $x$ which do not satisfy $\PP$ is an open set. The property $\PP$ therefore separates the set $X$ into two open complementary sets. Due to connectedness, one of these sets must be empty.\\

Certainly, there is a point $x$ in $X$ for which $\PP$ holds. Choose an arbitrary point $x$ and a path $\sigma: [0,1]\to Y$ from $z$ to $x$. Let $t$ be the infimum of the set $\sigma^{-1}(X)$. Since $X$ is closed in $Y$, we see that $\sigma(t) \in X$. So clearly, $t>0$. Let $x' = \sigma(t)$ and $\sigma' = \sigma|_{[0,t]}$. Then for any open $U$ around $x'$, the set $\sigma'^{-1}(U)$ is an open set in $[0,t]$ around $t$. Hence, there is some point $p = \sigma'(t-\epsilon)$ in $U$ and a path $\sigma|_{[0,t-\epsilon]}$ from $z$ to $p$ which does not intersect $X$. Since $x'$ satisfies $\PP$, we see that all of $X$ satisfies $\PP$. \\

For $x$ in $X$, choose a small chart $U$ around $x$ where $X\subset Y$ in the chart is $\R^{n-1}\subset \R^n$. Then $U-X$ has two connected components. Choose $z_1, z_2$ in different components. For any point $z$ outside $X$, there is some point in $U-X$ connected to $z$ by a path outside $X$. This point is connected (away from $X$) to either $z_1$ or $z_2$ depending on the component of $U-X$ it lies in. So every point $z$ outside $X$ is connected to either $z_1$ or $z_2$ via a path in $Y - X$. There are thus at most two connected components of $Y-X$.

\section*{(b)}

Consider the two points $z_1$ and $z_2$ from (a). In the chart $U$, there is clearly the straight line path (in $\R^n$) from $z_1$ and $z_2$ which intersects $X$ exactly once (and is transverse). Pulling this back via the chart yields an embedding $\sigma:[0,1] \to Y$ from $z_1$ to $z_2$ which is transverse to $X$ and intersects $X$ exactly once. Call this path $C$. Since $Y-X$ has only one component, we may find a different path $\sigma'$ from $z_1$ to $z_2$ which does not intersect $X$. Assume $\sigma'$ is an embedding (injective, immersion): See note at the end of the solution, since I feel this is slightly derailing. \\

Observe that $C\subset U$. Let $t$ be the infimum of $\sigma'^{-1}(C)$ and $z_1' = \sigma'(t)$. Let $s$ be the supremum of $\sigma^{-1}(C)$ and let $z_2' = \sigma'(s)$. These points are well defined since $C$ is closed. We chose these new points to avoid the intersection of the curves. Now $\sigma'' = \sigma'|_{[t,s]}$ is a curve, and $C'$ be the part of $C$ between $z_1'$ and $z_2'$. If $C''$ is the image of $\sigma''$, then $C'' \cup C'$ is a loop without self-intersection, so it is homeomorphic to $S^1$. We may perturb $C''\cup C'$  relative to $C'$ so that $C'' \cup C'$ is now diffeomorphic to $S^1$. Call this submanifold $Z$ (submanifold, since it is a smooth embedding of $S^1$). Then $Z\cap X$ is only one point and the intersection is transverse at that point, so $I_2(X,Z) = 1$.\\

\textit{Note:} Let $\sigma:[0,1]\to M$ be a path from $z_1\neq z_2$ in a manifold $M$. We want an injective map $\sigma':[0,1] \to M$ whose image is contained in the image of sigma. For any $z$ in the image of $\sigma$, define $I_z = (a,b)$ where $a$ and $b$ are the infimum and supremum of $\sigma^{-1}(z)$. Consider the set $U = \cup_z I_z$. Then $U$ is open and is therefore a union of (countably many) $(a_i,b_i)$ with $\sigma(a_i) = \sigma(b_i)$. We may paste $\sigma$ along the complements of these intervals to get a map $\sigma': [0,1] \to M$ satisfying the required properties.

After some googles, we get result path connected implies arc connected in Hausdorff spaces. All arguments work.

\section*{(c)}
Every compact $1$-dimensional manifold without boundary is diffeomorphic to disjoint copies of $S^1$. There must be only finitely many such copies because of compactness, since each connected component is open. \\

Let $Z$ be a connected closed $1$-dimensional submanifold of $Y$. We may perturb $Z$ so that $Z$ is still diffeomorphic to $S^1$ but $Z$ is now also transverse to $X$. We will show that the size of $Z \cap X$ is even. Consider $Z$ as a map $\rho: [0,1] \to Y$, with $\rho(0) = \rho(1) \not \in X$ (From transversality, say.), where $\rho: S^1 \to Y$ is an embedding. Let $t \in [0,1]$ for which $\rho(t) \in X$. In the chart where $X\subset Y$ becomes $\R^{n-1}\subset \R^n$, the transversality condition means that $\rho'(t)$ has non-zero last coordinate. It follows that for all $\epsilon > 0$ small enough, $\rho(t+\epsilon)$ lies in one component, while $\rho(t - \epsilon)$ lies in the other component.\\ 

Let $C_1, C_2$ be the two components of $Y-X$. Let $t_1 < t_2 < \dots < t_n$ be points in $[0,1]$ which correspond to the points in $Z\cap X$ under $\rho$. WLOG, assume $\rho(0)$ lies in $C_1$. Then the argument in the above paragraph can be summarized as \[\text{For all $n$, there is some small $\epsilon_n$ such that } \rho(t_n + \epsilon_n) \in C_{n+1}, \rho(t_n-\epsilon_n) \in C_n,\] where $C_i = C_{i\bmod 2}$. However, since $\rho(0) = \rho(1)$, we also want $\rho(t_n+\epsilon_n) \in C_1$, so $C_{n+1} = C_1$, which means $n$ must be even.\\

So for $Z$ diffeomorphic to $S^1$, we have $I_2(X,Z) = 0$. In general, $Z$ is diffeomorphic to a disjoint union of finitely many $S^1$, so $I_2(X,Z)$ is still $0$. The other direction was proved in (b).

\section*{(d)}

Consider $S^1 = \R\P^1 = [x_0:0:\dots:0:x_{n+1}]$ as a submanifold of $\R\P^{n+1}$. Then at the (only) intersection point $[1:0:\dots:0]$ of $\R\P^1$ and $\R\P^n$, we compute the tangent spaces in the standard chart to be $\R^n \times 0$ and $0\times 0 \times \dots \times \R$ as subspaces of $\R^{n+1}$. Clearly the intersection is transverse and $I_2(\R\P^n, \R\P^1) = 1$. Therefore, $\R\P^n$ does not split $\R\P^{n+1}$ into two components by virtue of (c). \\

Alternatively it is straightforward (from the definition of the chart map) to show that $\R\P^{n+1} = \R^{n+1} \sqcup \R\P^n$, and since $\R^{n+1}$ is connected, the result follows.


% ------------------------------------------------------------------------------

% ------------------------------------------------------------------------------
% Reference and Cited Works
% ------------------------------------------------------------------------------

% \bibliographystyle{IEEEtran}
% \bibliography{References.bib}

% ------------------------------------------------------------------------------

\end{document}