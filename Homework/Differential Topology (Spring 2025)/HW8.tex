\documentclass{amsart}

\usepackage{amsmath,amssymb,amsfonts,amscd}
\usepackage[all,cmtip]{xy}
\usepackage{enumerate}
\usepackage{ amssymb, latexsym, amsmath}
%\setcounter{MaxMatrixCols}{30}%
\usepackage{fullpage}
\usepackage{url}
\usepackage{hyperref}
\usepackage{ marvosym }
\usepackage{quiver}

\numberwithin{equation}{section}

%\usepackage{cjw-latex}

%\providecommand{\U}[1]{\protect\rule{.1in}{.1in}}

\theoremstyle{plain}
\newtheorem{theorem}{Theorem}[section]
\newtheorem*{thm}{Theorem}
\newtheorem{proposition}[theorem]{Proposition}
\newtheorem{lemma}[theorem]{Lemma}
\newtheorem{conjecture}[theorem]{Conjecture}
\newtheorem{corollary}[theorem]{Corollary}
\newtheorem{algorithm}[theorem]{Algorithm}
\newtheorem{axiom}[theorem]{Axiom}
\newtheorem{criterion}[theorem]{Criterion}
%\newtheorem{conjecture}[theorem]{Conjecture}
\newtheorem{wildconjecture}[theorem]{Wild Conjecture}


\theoremstyle{definition}
\newtheorem{definition}[theorem]{Definition}
\newtheorem*{dfn}{Definition}
\newtheorem{condition}[theorem]{Condition}
\newtheorem{example}[theorem]{Example}
\newtheorem{exercise}[theorem]{Exercise}
\newtheorem{notation}[theorem]{Notation}
\newtheorem{question}[theorem]{Question}
\newtheorem{problem}[theorem]{Problem}
\newtheorem{solution}[theorem]{Solution}




\theoremstyle{remark}
\newtheorem{remark}[theorem]{Remark}
\newtheorem{remarks}[theorem]{Remarks}
\newtheorem{summary}[theorem]{Summary}
\newtheorem{observation}[theorem]{Observation}
\newtheorem{conclusion}[theorem]{Conclusion}
\newtheorem{acknowledgement}[theorem]{Acknowledgement}
\newtheorem{case}[theorem]{Case}
\newtheorem{claim}[theorem]{Claim}



\makeatletter
\newcommand{\rmnum}[1]{\romannumeral #1}
\newcommand{\Rmnum}[1]{\expandafter\@slowromancap\romannumeral #1@}
\makeatother
\newcommand{\Aut}{\operatorname{Aut}}
\newcommand{\Ext}{\operatorname{Ext}}
\newcommand{\Tor}{\operatorname{Tor}}
\newcommand{\Z}{\mathbb{Z}}
\newcommand{\osum}{\oplus}
\newcommand{\aff}{\operatorname{Aff}}

\newcommand{\Id}{\operatorname{Id}}
\newcommand{\concat}{(\Id \to \Omega \Sigma)}
\newcommand{\eval}{(\Sigma\Omega \to \Id)}
\newcommand{\Q}{\mathbb{Q}}
\newcommand{\F}{\mathbb{F}}
\newcommand{\N}{\mathbb{N}}
\newcommand{\g}{\mathfrak{g}}
\newcommand{\n}{\mathfrak{n}}
\newcommand{\h}{\mathfrak{h}}
\newcommand{\p}{\mathfrak{p}}
\newcommand{\q}{\mathfrak{q}}
\newcommand{\m}{\mathfrak{m}}
\newcommand{\e}{\mathfrak{e}}
\newcommand{\f}{\mathfrak{f}}
\newcommand{\V}{\mathscr{V}}
\newcommand{\D}{\mathscr{D}}
\renewcommand{\S}{\mathscr{S}}
\newcommand{\Sch}{\mathscr{C}}
\newcommand{\Y}{\mathscr{Y}}
\renewcommand{\u}{\mathfrak{u}}
\newcommand{\set}[2]{ \left\{ {#1} \, \left| \, {#2} \right\}\right.}
\newcommand{\A}{\mathbb{A}}
\newcommand{\X}{\mathcal{X}}
\renewcommand{\Im}{\operatorname{Im}}
\newcommand{\T}{\mathscr{T}}
\newcommand{\HH}{\mathscr{H}}
\newcommand{\PGL}{\operatorname{PGL}}
\newcommand{\righthookarrow}{\hookrightarrow}
\newcommand{\sign}{\operatorname{sign}}
\renewcommand{\_}[2]{\underbrace{#1}_{#2}}
\renewcommand{\^}[2]{\overbrace{#1}_{#2}}
\newcommand{\curverightarrow}{\curvearrowright}

\newcommand{\z}{\mathfrak{z}}
\renewcommand{\sl}{\mathfrak{sl}}
\newcommand{\R}{\mathbb{R}}
\newcommand{\RP}{\mathbb{RP}}
\renewcommand{\P}{\mathbb{P}}
\newcommand{\C}{\mathbb{C}}
\renewcommand{\H}{\mathcal{H}}
\newcommand{\Ind}{\operatorname{Ind}}
\newcommand{\Tr}{\operatorname{Tr}}
\newcommand{\Orb}{\mathcal{O}}
\renewcommand{\k}{\mbox{\Fontauri k}}
\newcommand{\Ad}{\operatorname{Ad}}
\newcommand{\ad}{\operatorname{ad}}
\newcommand{\Hom}{\operatorname{Hom}}
\newcommand{\Ker}{\operatorname{Ker}}
\newcommand{\End}{\operatorname{End}}
\newcommand{\supp}{\operatorname{supp}}
\newcommand{\Stab}{\mbox{Stab}}
\newcommand{\Lie}{\operatorname{Lie}}
\newcommand{\GL}{\operatorname{GL}}
\newcommand{\SL}{\operatorname{SL}}
\newcommand{\Sp}{\operatorname{Sp}}
\newcommand{\Mp}{\operatorname{Mp}}
\renewcommand{\a}{\mathfrak{a}}
\newcommand{\Wh}{\operatorname{Wh}}
\newcommand{\Span}{\operatorname{Span}}
\newcommand{\WF}{\operatorname{WF}}
\newcommand{\AC}{\operatorname{AC}}
\newcommand{\Lin}{\operatorname{Lin}}
\newcommand{\Diff}{\operatorname{Diff}}
\newcommand{\Gen}{\operatorname{Gen\, Hom}}
\newcommand{\Her}{\operatorname{Her}}
\newcommand{\pivot}{&}
\newcommand{\sgn}{\operatorname{sgn}}
\newcommand{\id}{\operatorname{id}}
%\newcommand{C^{\infty}}{DS}
\newcommand{\vsp}{{\vspace{0.2in}}}



\title{Homework 8}

\author{Sudharshan K V}
\begin{document}
\maketitle

\section*{Problem 1}
Let $M$ be a simply connected manifold. Choose a point $p$ in $M$. Fixing an orientation of $T_pM$, we will assign orientations to the other points $q$ via paths from $p$. Given an orientation at a point, and a connected chart around the point, one can assign orientations to the tangent spaces at all points in the connected chart compatible with the given orientation. \\

Let $s:[0,1] \to M$ be a (continuous) path from $p$ to $q$. Cover the image of $s$ with finitely many (due to compactness) connected charts. For convenience, let us call a covering by connected charts a connected cover. Let $\delta$ be the Lebesgue number for the pullback of this covering to $[0,1]$. Choose a partition $t_1, \dots, t_n$ such that the width of the sub-intervals is smaller than $\delta$. Then the image of each $[t_i, t_{i+1}]$ lies entirely within a connected chart. We will say a partition is \emph{associated} to the open cover if the image of every subinterval lies inside one of the open sets of the cover. So starting from $s(t_1) = s(0) = p$, one can successively assign orientations at the points $s(t_1), \dots, s(t_n) = q$ through the connected charts. Therefore, one obtains an orientation on $T_q$ via the path $s$. \\

We will show that the assignment is well-defined. First, given an open cover and a partition, it is easy to see that the assignment is well-defined, for the assignment of $T_iM := T_{s(t_i)}M$ uniquely determines the orientation of $T_{i+1}M$ independent of the chart chosen from the cover. Let $U_i$ be a (finite) connected cover. We say a connected cover $V_j$ \emph{refines} the cover $U_i$ if for all $j$ there is some $i$ for which $V_j \subset U_i$. A partition $t_i'$ is said to \emph{refine} the partition $t_i$ if we have as sets $\{ t_i'\} \subset \{ t_i\}$. 
\setcounter{section}{1}
  \begin{lemma}
    Let $U_i$ be a connected cover and $t_i$ be a partition of $[0,1]$ associated to the cover $U_i$. Let $V_j$ be a (connected) refinement of $U_i$, and let $t_j'$ be a partition which refines $t_i$ and is also associated to $V_j$. Then the orientations of $q$ obtained from the pair $(U_i, t_i)$ and $(V_j, t_j')$ coincide.
  \end{lemma}
  \begin{proof}
    This is easy. Let $t'_j = t_i$. Suppose the orientations determined at $t_i$ in the two ways agree. The orientation at $s(t'_{j+1})$ determined by $s(t_j') = s(t_i)$ agrees with the orientation determined on $s([t_i,t_{i+1}])$ since they both lie in a connected chart. Continuing, one sees that the orientation at $t'_{j+m} = t_{i+1}$ is the same when obtained in both ways. The result follows.
  \end{proof}

  \begin{lemma}
    The orientation provided on $T_qM$ via a fixed path $s$ is independent of the open cover or the parition associated to it.
  \end{lemma}
  \begin{proof}
    Let $U_i$, $V_j$ be connected open covers of the image of $s$. Let $t_i$ and $t_j'$ be associated paritions. Consider all possible intersections $U_i \cap V_j$ and consider the connected components $W_k$ of these intersections. The $W_k$ form a connected open cover of the image of $s$. Make this cover a finite cover by compactness. Observe that one can choose a partition $t_k''$ associated to $W_k$ which also refines both $t_i$ and $t_j'$. By the previous lemma, the orientations assigned by $U_i$ and $V_j$ agree with the orientation assigned by $W_k$. 
  \end{proof}

  Finally, we will show that this orientation is independent of the path $s$. This uses the fact that $M$ is simply connnected. Let $s_0, s_1$ be two paths from $p$ to $q$. Let $s_t$ be a homotopy from $s_0$ to $s_1$. Cover the image of the homotopy (i.e. image of all the $s_t$) by (finitely many) connected open charts. Let $U_i$ be the pullback of this cover to the unit square. Let $\delta$ be the Lebesgue number of the cover $U_i$. Choose a partition $t_i$ of $[0,1]$ with width $\delta/2$. The partition $t_i$ also suits the paths $s_{t\pm\epsilon}$ for $\epsilon << \delta$, and furthermore, each of the rectangles $[t_i,t_{i+1}] \times [t-\epsilon, t+\epsilon]$ lies inside one of the charts $U_i$. Consider these charts which cover these rectangles. These charts correspond to a connected open cover of the image of $s_r$ for any $r$ between $t\pm \epsilon$. One can now see that the orientation obtained on $q$ is the same for all such $r$, since the choice of cover is the same. (More precisely, the orientation of $s_r(t_i)$ agrees (in the connected chart containing both) with the orientation of $s_t(t_i)$ for all $i$.)\\

  It follows that the assignment of orientation through $s_t$ for each $t$ is locally constant in $t$. Due to $[0,1]$ being connected, the orientations on $T_qM$ obtained through $s_0, s_1$ must agree. Therefore the process assigns a well-defined orientation to $T_q$ at each point $q$ in $M$. It is easy to see that the orientations are compatible: Let $q'$ be a point inside an open connected chart around $q$. Let $s$ be a path from $p$ to $q$ which passes through $q'$. (Such a path always exists.) The orientation assigned at $q$ must be compatible with the orientation assigned at $q'$ through the same path. So the orientation assigned by the process is smoothly varying (i.e. locally constant), meaning $M$ is orientable.
  
\section*{Problem 2}
Let $m,n$ be the dimensions of $X,Z$ respectively. The dimension of $Y$ is $m+n$. \\

(a) A homotopy of $f$ to $f'$ and a homotopy of $g$ to $g'$ induce a homotopy of $f\times g$ to $f'\times g'$ in the obvious way. So $I(f,g)$ is unaltered by homotopy of either $f$ or $g$. \\

(b) Let $g$ be the inclusion $Z\hookrightarrow Y$. First observe the correspondence between $(f\times g)^{1}(\Delta)$ and $f^{-1}(Z)$, given by $(x,z) \mapsto x$.

Next, we show that transversality of $f$ and $Z$ is the equivalent to the transversality of $f\times g$ and $\Delta$. Let $W$ be the image of $Df_x$, and let $z = f(x)$. Suppose now that $f$ is transverse to $Z$. Then, $W + T_zZ = T_zY$. For $(a,b) \in T_{(f(x),z)}Y\times Y = T_zY \oplus T_zY$, let $u = \frac12 (a+b)$ and choose $w, v$ in $W, T_zZ$ such that $w-v = a-b$ (from transversality). Then we have \[(w,v) + (u-\tfrac12 (w+v), u -\tfrac12 (w+v)) = (u + \tfrac12 (w-v), u - \tfrac12 (w-v)) = (a,b),\] which is transversality of $f\times g$ and $\Delta$ at $(x,z)$.

Conversely, suppose $f\times g$ is transverse to $\Delta$. Then, for any $a$ in $T_zY$, one has $w, v$ in $W, T_zZ$ for which $(w,-v) + (u,u) = (\frac a2,-\frac a2)$. This is equivalent to
\begin{align*}
  w + u & = \frac a 2\\
  -v + u & = -\frac a 2.\\
\end{align*}
Subtracting shows $w+v = a$, which is transversality of $f,Z$. Combined with the correspondence of intersection points, one has the equivalence of transversality.\\

Next we will show equality of intersection numbers. We have the equality of vector spaces \[T_xX \oplus T_zZ = T_yY.\] Let $(x_1\dots x_m)$, $(z_1 \dots z_n)$ be bases of $X,Z$. Suppose that the basis of $Z$ is oriented positively and that the basis $(x_1, \dots, x_m, z_1, \dots ,z_n)$ is positively oriented in $T_yY$. The orientation of the $x$ is decided by whether $(x_1, \dots ,x_m)$ is a positive or a negative basis of $T_xX$. \\

Now consider the equality obtained from transversality of $f\times g$: \[T_xX \oplus T_zZ \oplus T_{(z,z)}Z = T_zY \oplus T_zY.\] One computes the change of basis matrix for \[\left( (x_1,0,0) \dots, (x_m,0,0),(0,0,z_1), \dots ,(0,0,z_n), (x_1,x_1), \dots (z_n,z_n)\right ) \text{ wrt } \left( (x_1, 0), \dots, (z_n,0), (0,x_1), \dots , (0,z_n) \right ) \,\,\,\,\, \star \] to be
\[\begin{pmatrix}
  I_m & 0 & I_m & 0\\
  0 & 0 & 0 & I_n\\
  0 & 0 & I_m & 0\\
  0 & I_n & 0 & I_n
\end{pmatrix}.\]
Using row operations (it takes $n$ swaps to make this upper triangular), one finds the determinant of this matrix to be $(-1)^n$. Furthermore, all the bases above except that of $T_xX$ are positive, so the sign given to the point $(x,z)$ is the same as $(-1)^n$ times the sign of $x$ when given the orientation on the preimage $f^{-1}(Z)$. The factor $(-1)^n$ is offset by the definition of $I(f,g) = (-1)^n I(f\times g, \Delta)$, so it follows that $I(f,Z) = I(f, g = i_Z)$. \\

(c) Use the same notation as above, identifying $T_zZ$ with its image under $Dg_z$. Let $f(x) = g(z) = y$. We have an equality of vector spaces \[T_xX \oplus T_zZ \oplus T_{(y,y)}\Delta = T_yY\oplus T_yY.\] Assume that $(x_1,\dots, z_n)$ is still a positive basis of $Y$. One finds that swapping $X,Z$ modifies the change of basis matrix to \[
  \begin{pmatrix}
    0 & 0 & I_m & 0\\
    I_n & 0 & 0 & I_n\\
    0 & I_m & I_m & 0\\
    0 & 0 & 0 & I_n.
  \end{pmatrix}
\]
Swap the middle two block rows and the first and third block columns. The resulting matrix is upper triangular. This shows that the determinant of this change of basis is $(-1)^m (-1)^{mn}$. It follows that \[(-1)^{mn}(-1)^m I(g\times f, \Delta)= I(f,g),\] after which it is immediate that $I(f,g) = (-1)^{mn} I(g,f)$. 
\section*{Problem 3}

(a) Let $p\in Z$ be a regular value of $g$. Then $g^{-1}(p)$ is a discrete set of size $\deg g$. Perturb $f$ to map $f'$ so that $f'$ is regular at each point of $g^{-1}(p)$. One can do this because $g^{-1}(p)$ is finite and closed in $Y$. The homotopy of $f$ induces a homotopy of $g\circ f$ to $g \circ f'$ in the obvious way, and $p$ is a regular value of $g \circ f'$ by construction. It follows that \[\deg (g\circ f) = \deg (g\circ f') = \deg f' \cdot \deg g = \deg f \cdot \deg g.\] We use the fact that the degree is the size of the preimage of any regular value to get the equality in the middle.\\

(b) The Euler characteristic is the Lefschetz number of the identity. Let $f$ and $g$ be Lefschetz maps homotopic to the identity maps on $X,Y$. The map $f\times g$ is homotopic to the identity on $X\times Y$, and one has that the fixed points of $f\times g$ are of the form $(x,y)$ where $x$ is fixed by $f$ and $y$ is fixed by $g$. The equality $L(f\times g) = L(f)L(g)$ follows, and this is exactly the multiplicativity of the Euler characteristic.\\

(c) Let $G$ be a compact Lie group of positive dimension. We will show that $G$ admits a non-vanishing vector field. Indeed, let $\g = T_eG$ be the Lie algebra of $G$. Choose a vector $v \in \g$ which is non-zero. Recall that $T_gG = gT_eG$ for $g\in G$. Then the assignment $\vec v: g \mapsto gv$ is therefore a non-vanishing vector field on $G$. It follows that the Euler characteristic of $G$ must be $0$.

\section*{Problem 4}

(a) If two of the $\lambda_i$ are equal, say $\lambda_1 = \lambda_2$, then all points $[x:1:0:0]$ are fixed by $f$. These points are not isolated. If no two are equal there are exactly four fixed points: $p_0 = [1:0:0:0], p_1 = [0:1:0:0], \dots$. These points are isolated, and in the charts around the points $p_i$, $f$ is linear and the derivative is given by the diagonal matrix with diagonal entries $\lambda_j/\lambda_i$ for $j\neq i$. Since no two of the $\lambda_i$ are equal, $1$ is not an eigenvalue of the derivative of $f$. So $f$ is Lefschetz iff no two of the $\lambda_i$ are equal.\\

(b) Continuing the previous discussion, the determinant of $Df_{p_i} - \id$ has the same sign as $\lambda_i^3\prod_j (\lambda_j - \lambda_i)$. The sign is also the sign given to the fixed point $p_i$: \[\text{Orientation of $p_i$} = \sgn\left (\lambda_i\prod_j (\lambda_j - \lambda_i)\right).\] Since the expressions are symmetric, we may assume that $\lambda_0 < \dots < \lambda_3$. Depending on the signs of the $\lambda_i$, we calculate the Lefschetz numbers as follows: (assuming increasing $\lambda_i$.) If $0<\lambda_0$ the points get orientations $(-,+,-,+)$, and the Lefschetz number is $0$. If $\lambda_0 < 0 < \lambda_1$ then the points get orientations $(+, +, - , +)$ so the Lefschetz number is $2$. If $\lambda_1 < 0 <\lambda_2$, the points get orientations $(+,-,-,+)$ and the Lefschetz number of $f$ is $0$. The other cases can be obtained similarly. If $0$ is between $\lambda_i, \lambda_{i+1}$, then the Lefschetz number of $f$ is $2 (i-1\bmod 2)$. (Put $\lambda_{-1} = -\infty, \lambda_4 = \infty$.)\\

(c) The identity map is easily seen to be homotopic to the map $f$ when $\lambda_i = 2^i$, via the homotopy \[[x_0:\dots:x_3] \mapsto [x_0:(1+t)x_1:(1+(2^2-1)t)x_2:(1+(2^3-1)t)x_3].\] Using the previous calculation, the Lefschetz number of such a map is $0$, hence the Euler characteristic of $\R\P^3$ is $0$.\\

(d) The maps corresponding to $(\lambda_i) = (1,2,3,4)$ and $(\lambda_i) = (-1,2,3,4)$ have different Lefschetz numbers, so they are not homotopic.

\section*{Problem 5}

(a) If $f(z) = z$, then $z^m = 0$. The only fixed point of $f$ is thus $0$ and the first result follows. The derivative of $f$ at $0$ is $1$ (in complex numbers, which is the identity matrix over reals) unless $m = 1$, whence the derivative is twice the identity. Therefore $0$ is not Lefschetz unless $m = 1$.\\

(b) Use (c). The sum of local Lefschetz numbers is the global Lefschetz number.\\

(c) Consider the perturbation $f'$ of $f$ to $z\mapsto f(z) + \epsilon$. Observe that $z$ is a fixed point of $f'$ if and only if it satisfies $z^m = -\epsilon \neq 0$, and this polynomial has $m$ distinct solutions in $\C$. Furthermore, the derivative of $f'$ at this point is $mz^{m-1} + 1$, which is never equal to $1$ since $z$ is non-zero. The complex derivative not equal to $1$ is the same as the real derivative having no eigenvalue $1$.

\section*{Problem 6}
See that $f$ is not Lefschetz since it has non isolated fixed points: all points $[r:1]$ for $r\in \R$ are fixed by $f$. However, $f:[z_1:z_2] \mapsto [\overline z_1: \overline z_2]$ is homotopic to the map $[z_1:z_2] \mapsto [2\overline z_1: \overline z_2]$. The new map has only two fixed points $[0:1]$ and $[1:0]$. These points are isolated, and in charts, the maps are given by $(x,y) \mapsto (2x, -2y)$ and $(x,y) \mapsto (\frac x2,-\frac y2)$ (after converting a complex map to a real map). It follows that the matrix of the differentials of these maps are just \[
  \begin{pmatrix}
    2 & 0 \\ 0 & -2
  \end{pmatrix} \text{ and }
  \begin{pmatrix}
    \frac12 & 0 \\ 0 & -\frac12
  \end{pmatrix}.
\]
The map $f'$ is therefore Lefschetz, and the local Lefschetz numbers at these two points are $1$ and $-1$ respectively. It follows that $L(f) = 0$. 

\section*{Problem 7}
For $n$ even, the Euler characteristic of $S^n$ is non-zero, so $S^n$ does not admit a non-vanishing vector field.\\

For $n$ odd, consider $S^n \subset \R^{n+1}$. The tangent space $T_xS^n$ can be identified with the subspace of $\R^{n+1}$ which is orthogonal to $x$. Therefore the assignment \[(x_1, \dots, x_{n+1}) \mapsto (x_2, -x_1, x_4, -x_3, \dots, x_{n+1}, -x_n)\] (Recall that $n$ is odd, so this definition makes sense.) is a vector field which is non-vanishing everywhere. 
y
\end{document}
