\documentclass{amsart}

\usepackage{amsmath,amssymb,amsfonts,amscd}
\usepackage[all,cmtip]{xy}
\usepackage{enumerate}
\usepackage{ amssymb, latexsym, amsmath}
%\setcounter{MaxMatrixCols}{30}%
\usepackage{fullpage}
\usepackage{url}
\usepackage{hyperref}
\usepackage{ marvosym }
\usepackage{quiver}

\numberwithin{equation}{section}

%\usepackage{cjw-latex}

%\providecommand{\U}[1]{\protect\rule{.1in}{.1in}}

\theoremstyle{plain}
\newtheorem{theorem}{Theorem}[section]
\newtheorem*{thm}{Theorem}
\newtheorem{proposition}[theorem]{Proposition}
\newtheorem{lemma}[theorem]{Lemma}
\newtheorem{conjecture}[theorem]{Conjecture}
\newtheorem{corollary}[theorem]{Corollary}
\newtheorem{algorithm}[theorem]{Algorithm}
\newtheorem{axiom}[theorem]{Axiom}
\newtheorem{criterion}[theorem]{Criterion}
%\newtheorem{conjecture}[theorem]{Conjecture}
\newtheorem{wildconjecture}[theorem]{Wild Conjecture}


\theoremstyle{definition}
\newtheorem{definition}[theorem]{Definition}
\newtheorem*{dfn}{Definition}
\newtheorem{condition}[theorem]{Condition}
\newtheorem{example}[theorem]{Example}
\newtheorem{exercise}[theorem]{Exercise}
\newtheorem{notation}[theorem]{Notation}
\newtheorem{question}[theorem]{Question}
\newtheorem{problem}[theorem]{Problem}
\newtheorem{solution}[theorem]{Solution}




\theoremstyle{remark}
\newtheorem{remark}[theorem]{Remark}
\newtheorem{remarks}[theorem]{Remarks}
\newtheorem{summary}[theorem]{Summary}
\newtheorem{observation}[theorem]{Observation}
\newtheorem{conclusion}[theorem]{Conclusion}
\newtheorem{acknowledgement}[theorem]{Acknowledgement}
\newtheorem{case}[theorem]{Case}
\newtheorem{claim}[theorem]{Claim}



\makeatletter
\newcommand{\rmnum}[1]{\romannumeral #1}
\newcommand{\Rmnum}[1]{\expandafter\@slowromancap\romannumeral #1@}
\makeatother
\newcommand{\Aut}{\operatorname{Aut}}
\newcommand{\Ext}{\operatorname{Ext}}
\newcommand{\Tor}{\operatorname{Tor}}
\newcommand{\Z}{\mathbb{Z}}
\newcommand{\osum}{\oplus}
\newcommand{\aff}{\operatorname{Aff}}

\newcommand{\Id}{\operatorname{Id}}
\newcommand{\concat}{(\Id \to \Omega \Sigma)}
\newcommand{\eval}{(\Sigma\Omega \to \Id)}
\newcommand{\Q}{\mathbb{Q}}
\newcommand{\F}{\mathbb{F}}
\newcommand{\N}{\mathbb{N}}
\newcommand{\g}{\mathfrak{g}}
\newcommand{\n}{\mathfrak{n}}
\newcommand{\h}{\mathfrak{h}}
\newcommand{\p}{\mathfrak{p}}
\newcommand{\q}{\mathfrak{q}}
\newcommand{\m}{\mathfrak{m}}
\newcommand{\e}{\mathfrak{e}}
\newcommand{\f}{\mathfrak{f}}
\newcommand{\V}{\mathscr{V}}
\newcommand{\D}{\mathscr{D}}
\renewcommand{\S}{\mathscr{S}}
\newcommand{\Sch}{\mathscr{C}}
\newcommand{\Y}{\mathscr{Y}}
\renewcommand{\u}{\mathfrak{u}}
\newcommand{\set}[2]{ \left\{ {#1} \, \left| \, {#2} \right\}\right.}
\newcommand{\A}{\mathbb{A}}
\newcommand{\X}{\mathcal{X}}
\renewcommand{\Im}{\operatorname{Im}}
\newcommand{\T}{\mathscr{T}}
\newcommand{\HH}{\mathscr{H}}
\newcommand{\PGL}{\operatorname{PGL}}
\newcommand{\righthookarrow}{\hookrightarrow}
\newcommand{\sign}{\operatorname{sign}}
\renewcommand{\_}[2]{\underbrace{#1}_{#2}}
\renewcommand{\^}[2]{\overbrace{#1}_{#2}}
\newcommand{\curverightarrow}{\curvearrowright}

\newcommand{\z}{\mathfrak{z}}
\renewcommand{\sl}{\mathfrak{sl}}
\newcommand{\R}{\mathbb{R}}
\newcommand{\RP}{\mathbb{RP}}
\renewcommand{\P}{\mathbb{P}}
\newcommand{\C}{\mathbb{C}}
\renewcommand{\H}{\mathcal{H}}
\newcommand{\Ind}{\operatorname{Ind}}
\newcommand{\Tr}{\operatorname{Tr}}
\newcommand{\Orb}{\mathcal{O}}
\renewcommand{\k}{\mbox{\Fontauri k}}
\newcommand{\Ad}{\operatorname{Ad}}
\newcommand{\ad}{\operatorname{ad}}
\newcommand{\Hom}{\operatorname{Hom}}
\newcommand{\Ker}{\operatorname{Ker}}
\newcommand{\End}{\operatorname{End}}
\newcommand{\supp}{\operatorname{supp}}
\newcommand{\Stab}{\mbox{Stab}}
\newcommand{\Lie}{\operatorname{Lie}}
\newcommand{\GL}{\operatorname{GL}}
\newcommand{\SL}{\operatorname{SL}}
\newcommand{\Sp}{\operatorname{Sp}}
\newcommand{\Mp}{\operatorname{Mp}}
\renewcommand{\a}{\mathfrak{a}}
\newcommand{\Wh}{\operatorname{Wh}}
\newcommand{\Span}{\operatorname{Span}}
\newcommand{\WF}{\operatorname{WF}}
\newcommand{\AC}{\operatorname{AC}}
\newcommand{\Lin}{\operatorname{Lin}}
\newcommand{\Diff}{\operatorname{Diff}}
\newcommand{\Gen}{\operatorname{Gen\, Hom}}
\newcommand{\Her}{\operatorname{Her}}
\newcommand{\pivot}{&}
\newcommand{\sgn}{\operatorname{sgn}}
%\newcommand{C^{\infty}}{DS}
\newcommand{\vsp}{{\vspace{0.2in}}}



\title{Homework 8}

\author{Sudharshan K V}
\begin{document}
\maketitle

\section*{Problem 3}

(a) Let $p\in Z$ be a regular value of $g$. Then $g^{-1}(p)$ is a discrete set of size $\deg g$. Perturb $f$ to map $f'$ so that $f'$ is regular at each point of $g^{-1}(p)$. One can do this because $g^{-1}(p)$ is finite and closed in $Y$. The homotopy of $f$ induces a homotopy of $g\circ f$ to $g \circ f'$ in the obvious way, and $p$ is a regular value of $g \circ f'$ by construction. It follows that \[\deg (g\circ f) = \deg (g\circ f') = \deg f' \cdot \deg g = \deg f \cdot \deg g.\] We use the fact that the degree is the size of the preimage of any regular value to get the equality in the middle.\\

(b) The Euler characteristic is the Lefschetz number of the identity. Let $f$ and $g$ be Lefschetz maps homotopic to the identity maps on $X,Y$. The map $f\times g$ is homotopic to the identity on $X\times Y$, and one has that the fixed points of $f\times g$ are of the form $(x,y)$ where $x$ is fixed by $f$ and $y$ is fixed by $g$. The equality $L(f\times g) = L(f)L(g)$ follows, and this is exactly the multiplicativity of the Euler characteristic.\\

(c) Let $G$ be a compact Lie group of positive dimension. We will show that $G$ admits a non-vanishing vector field. Indeed, let $\g = T_eG$ be the Lie algebra of $G$. Choose a vector $v \in \g$ which is non-zero. Recall that $T_gG = gT_eG$ for $g\in G$. Then the assignment $\vec v: g \mapsto gv$ is therefore a non-vanishing vector field on $G$. It follows that the Euler characteristic of $G$ must be $0$.

\section*{Problem 4}

(a) If two of the $\lambda_i$ are equal, say $\lambda_1 = \lambda_2$, then all points $[x:1:0:0]$ are fixed by $f$. These points are not isolated. If no two are equal there are exactly four fixed points: $p_0 = [1:0:0:0], p_1 = [0:1:0:0], \dots$. These points are isolated, and in the charts around the points $p_i$, $f$ is linear and the derivative is given by the diagonal matrix with diagonal entries $\lambda_j/\lambda_i$ for $j\neq i$. Since no two of the $\lambda_i$ are equal, $1$ is not an eigenvalue of the derivative of $f$. So $f$ is Lefschetz iff no two of the $\lambda_i$ are equal.\\

(b) Continuing the previous discussion, the determinant of $Df_{p_i} - \id$ has the same sign as $\lambda_i^3\prod_j (\lambda_j - \lambda_i)$. The sign is also the sign given to the fixed point $p_i$: \[\text{Orientation of $p_i$} = \sgn(\lambda_i^3)\prod_j (\lambda_j - \lambda_i)\]. Since the expressions are symmetric, we may assume that $\lambda_0 < \dots < \lambda_3$. Depending on the signs of the $\lambda_i$, we calculate the Lefschetz numbers as follows: (assuming increasing $\lambda_i$.) If $0<\lambda_0$ the points get orientations $(-,+,-,+)$, and the Lefschetz number is $0$. If $\lambda_0 < 0 < \lambda_1$ then the points get orientations $(+, +, - , +)$ so the Lefschetz number is $2$. If $\lambda_1 < 0 <\lambda_2$, the points get orientations $(+,-,-,+)$ and the Lefschetz number of $f$ is $0$. The other cases can be obtained similarly. If $0$ is between $\lambda_i, \lambda_{i+1}$, then the Lefschetz number of $f$ is $2 (i-1\bmod 2)$. (Put $\lambda_{-1} = -\infty, \lambda_4 = \infty$.)\\

(c) The identity map is easily seen to be homotopic to the map $f$ when $\lambda_i = 2^i$, via the homotopy \[[x_0:\dots:x_3] \mapsto [x_0:(1+t)x_1:(1+(2^2-1)tx_2:(1+(2^3-1)t)x_3].\] Using the previous calculation, the Lefschetz number of such a map is $0$, hence the Euler characteristic of $\R\P^3$ is $0$.\\

(d) The maps corresponding to $(\lambda_i) = (1,2,3,4)$ and $(\lambda_i) = (-1,2,3,4)$ have different Lefschetz numbers, so they are not homotopic.

\section*{Problem 5}

(a) If $f(z) = z$, then $z^m = 0$. The only fixed point of $f$ is thus $0$ and the first result follows. The derivative of $f$ at $0$ is $1$ (in complex numbers, which is the identity matrix over reals) unless $m = 1$, whence the derivative is twice the identity. Therefore $0$ is not Lefschetz unless $m = 1$.\\

(b) Use (c). The sum of local Lefschetz numbers is the global Lefschetz number.\\

(c) Consider the perturbation $f'$ of $f$ to $z\mapsto f(z) + \epsilon$. Observe that $z$ is a fixed point of $f'$ if and only if it satisfies $z^m = -\epsilon \neq 0$, and this polynomial has $m$ distinct solutions in $\C$. Furthermore, the derivative of $f'$ at this point is $mz^{m-1} + 1$, which is never equal to $1$ since $z$ is non-zero. The complex derivative not equal to $1$ is the same as the real derivative having no eigenvalue $1$.

\section*{Problem 6}
See that $f$ is not Lefschetz since it has non isolated fixed points: all points $[r:1]$ for $r\in \R$ are fixed by $f$. However, $f:[z_1:z_2] \mapsto [\overline z_1: \overline z_2]$ is homotopic to the map $[z_1:z_2] \mapsto [2\overline z_1: \overline z_2]$. The new map has only two fixed points $[0:1]$ and $[1:0]$. These points are isolated, and in charts, the maps are given by $(x,y) \mapsto (2x, -2y)$ and $(x,y) \mapsto (\frac x2,-\frac y2)$ (after converting a complex map to a real map). It follows that the matrix of the differentials of these maps are just \[
  \begin{pmatrix}
    2 & 0 \\ 0 & -2
  \end{pmatrix} \text{ and }
  \begin{pmatrix}
    \frac12 & 0 \\ 0 & -\frac12
  \end{pmatrix}.
\]
The map $f'$ is therefore Lefschetz, and the local Lefschetz numbers at these two points are $1$ and $-1$ respectively. It follows that $L(f) = 0$. 

\section*{Problem 7}
For $n$ even, the Euler characteristic of $S^n$ is non-zero, so $S^n$ does not admit a non-vanishing vector field.\\

For $n$ odd, consider $S^n \subset \R^{n+1}$. The tangent space $T_xS^n$ can be identified with the subspace of $\R^{n+1}$ which is orthogonal to $x$. Therefore the assignment \[(x_1, \dots, x_{n+1}) \mapsto (x_2, -x_1, x_4, -x_3, \dots, x_{n+1}, -x_n)\] (Recall that $n$ is odd, so this definition makes sense.) is a vector field which is non-vanishing everywhere. 
y
\end{document}
