\documentclass{amsart}

\usepackage{amsmath,amssymb,amsfonts,amscd}
\usepackage[all,cmtip]{xy}
\usepackage{enumerate}
\usepackage{ amssymb, latexsym, amsmath}
%\setcounter{MaxMatrixCols}{30}%
\usepackage{fullpage}
\usepackage{url}
\usepackage{hyperref}
\usepackage{ marvosym }
\usepackage{quiver}



\numberwithin{equation}{section}

%\usepackage{cjw-latex}

%\providecommand{\U}[1]{\protect\rule{.1in}{.1in}}

\theoremstyle{plain}
\newtheorem{theorem}{Theorem}[section]
\newtheorem*{thm}{Theorem}
\newtheorem{proposition}[theorem]{Proposition}
\newtheorem{lemma}[theorem]{Lemma}
\newtheorem{conjecture}[theorem]{Conjecture}
\newtheorem{corollary}[theorem]{Corollary}
\newtheorem{algorithm}[theorem]{Algorithm}
\newtheorem{axiom}[theorem]{Axiom}
\newtheorem{criterion}[theorem]{Criterion}
%\newtheorem{conjecture}[theorem]{Conjecture}
\newtheorem{wildconjecture}[theorem]{Wild Conjecture}


\theoremstyle{definition}
\newtheorem{definition}[theorem]{Definition}
\newtheorem*{dfn}{Definition}
\newtheorem{condition}[theorem]{Condition}
\newtheorem{example}[theorem]{Example}
\newtheorem{exercise}[theorem]{Exercise}
\newtheorem{notation}[theorem]{Notation}
\newtheorem{question}[theorem]{Question}
\newtheorem{problem}[theorem]{Problem}
\newtheorem{solution}[theorem]{Solution}




\theoremstyle{remark}
\newtheorem{remark}[theorem]{Remark}
\newtheorem{remarks}[theorem]{Remarks}
\newtheorem{summary}[theorem]{Summary}
\newtheorem{observation}[theorem]{Observation}
\newtheorem{conclusion}[theorem]{Conclusion}
\newtheorem{acknowledgement}[theorem]{Acknowledgement}
\newtheorem{case}[theorem]{Case}
\newtheorem{claim}[theorem]{Claim}



\makeatletter
\newcommand{\rmnum}[1]{\romannumeral #1}
\newcommand{\Rmnum}[1]{\expandafter\@slowromancap\romannumeral #1@}
\makeatother
\newcommand{\Aut}{\operatorname{Aut}}
\newcommand{\Ext}{\operatorname{Ext}}
\newcommand{\Tor}{\operatorname{Tor}}
\newcommand{\Z}{\mathbb{Z}}
\newcommand{\osum}{\oplus}
\newcommand{\aff}{\operatorname{Aff}}

\newcommand{\Id}{\operatorname{Id}}
\newcommand{\concat}{(\Id \to \Omega \Sigma)}
\newcommand{\eval}{(\Sigma\Omega \to \Id)}
\newcommand{\Q}{\mathbb{Q}}
\newcommand{\F}{\mathbb{F}}
\newcommand{\N}{\mathbb{N}}
\newcommand{\g}{\mathfrak{g}}
\newcommand{\n}{\mathfrak{n}}
\newcommand{\h}{\mathfrak{h}}
\newcommand{\p}{\mathfrak{p}}
\newcommand{\q}{\mathfrak{q}}
\newcommand{\m}{\mathfrak{m}}
\newcommand{\e}{\mathfrak{e}}
\newcommand{\f}{\mathfrak{f}}
\newcommand{\V}{\mathscr{V}}
\newcommand{\D}{\mathscr{D}}
\renewcommand{\S}{\mathscr{S}}
\newcommand{\Sch}{\mathscr{C}}
\newcommand{\Y}{\mathscr{Y}}
\renewcommand{\u}{\mathfrak{u}}
\newcommand{\set}[2]{ \left\{ {#1} \, \left| \, {#2} \right\}\right.}
\newcommand{\A}{\mathbb{A}}
\newcommand{\X}{\mathcal{X}}
\renewcommand{\Im}{\operatorname{Im}}
\newcommand{\T}{\mathscr{T}}
\newcommand{\HH}{\mathscr{H}}
\newcommand{\PGL}{\operatorname{PGL}}
\newcommand{\righthookarrow}{\hookrightarrow}
\newcommand{\sign}{\operatorname{sign}}
\renewcommand{\_}[2]{\underbrace{#1}_{#2}}
\renewcommand{\^}[2]{\overbrace{#1}_{#2}}
\newcommand{\curverightarrow}{\curvearrowright}

\newcommand{\z}{\mathfrak{z}}
\renewcommand{\sl}{\mathfrak{sl}}
\newcommand{\R}{\mathbb{R}}
\newcommand{\RP}{\mathbb{RP}}
\renewcommand{\P}{\mathbb{P}}
\newcommand{\C}{\mathbb{C}}
\renewcommand{\H}{\mathbb{H}}
\newcommand{\Ind}{\operatorname{Ind}}
\newcommand{\Tr}{\operatorname{Tr}}
\newcommand{\Orb}{\mathcal{O}}
\renewcommand{\k}{\mbox{\Fontauri k}}
\newcommand{\Ad}{\operatorname{Ad}}
\newcommand{\ad}{\operatorname{ad}}
\newcommand{\Hom}{\operatorname{Hom}}
\newcommand{\Ker}{\operatorname{Ker}}
\newcommand{\End}{\operatorname{End}}
\newcommand{\supp}{\operatorname{supp}}
\newcommand{\Stab}{\mbox{Stab}}
\newcommand{\Lie}{\operatorname{Lie}}
\newcommand{\GL}{\operatorname{GL}}
\newcommand{\SL}{\operatorname{SL}}
\newcommand{\Sp}{\operatorname{Sp}}
\newcommand{\Mp}{\operatorname{Mp}}
\renewcommand{\a}{\mathfrak{a}}
\newcommand{\Wh}{\operatorname{Wh}}
\newcommand{\Span}{\operatorname{Span}}
\newcommand{\WF}{\operatorname{WF}}
\newcommand{\AC}{\operatorname{AC}}
\newcommand{\Lin}{\operatorname{Lin}}
\newcommand{\Diff}{\operatorname{Diff}}
\newcommand{\Gen}{\operatorname{Gen\, Hom}}
\newcommand{\Her}{\operatorname{Her}}
\newcommand{\pivot}{&}
\newcommand{\sgn}{\operatorname{sgn}}
%\newcommand{C^{\infty}}{DS}
\newcommand{\vsp}{{\vspace{0.2in}}}



\title{Homework 9}

\author{Sudharshan K V}
\begin{document}
\maketitle

\section*{Problem 1}
The topological manifold structure on $TX$ is given by the charts $\pi^{-1}(U)$ where $U$ is a chart on $X$. The homeomorphism to Euclidean space is given by the maps $h$. \\

Let $p$ in $X$, and let $U$, and $V$ be charts around $p$. We compute the transition map between charts $\pi^{-1}(U)$ and $\pi^{-1}(V)$. Let $f$ denote the transition map between $U$ and $V$ (i.e., from $\varphi_U(U\cap V)$ to $\varphi_V(U\cap V)$). For each $q$ in $U \cap V$, the charts $\varphi_U, \varphi_V$ give (two) isomorphisms of $T_qX$ to $\R^n$. The transition map between these isomorphisms is $Df_q$.

Now for $q$ in $U \cap V$ and a vector $v \in T_q$, we see that the transition map between the two (topological) charts $h$ wrt $U$ and $V$ takes \[(\varphi_U(q), D\varphi_U(v)) \mapsto (\varphi_V(q), D\varphi_V(v)).\] One therefore computes the transition map to be $(a, w) \mapsto (f(a), Df_a(w))$, which is clearly smooth. (The transition maps also induce isomorphisms of vector spaces at each fiber and hence $TX$ is a vector bundle).

\section*{Problem 2}
(a) The transition map in complex coordinates between the charts is given by $z \mapsto \frac1z$. The derivative of this map at $z$ is $-\frac1{z^2}$. Restricting the derivative to the points $z \in \C$ for which $|z| = 1$, one sees that the absolute value of the complex derivative is $1$. The complex absolute value of the derivative is equal to $MM^T$, where $M$ is the real derivative (see Midterm, Problem 4). This means that the image of $S^1$ under the derivative map to $\GL_2(\R)$ lands in $SO_2(\R)$ ($\det M>0$). Note that $SO_2$ is diffeomorphic to $S^1$ as Lie groups via the map \[
  \begin{pmatrix}
    a & -b\\
    b & a
  \end{pmatrix} \mapsto a+ib.\]
The composition of the map $z \mapsto Df_z$ and the above isomorphism is just $z \mapsto -\frac1{z^2}$. Consider the fixed point $-1$ of $S^1$ under this transformation. One can immediately see that this map represents multiplication by $2$ on the fundamental group $\Z$ with base point $-1$. \\

(b) Consider the $\R$-algebra of quarternion group $\H = \R[1,i,j,k]$, with the standard multiplication laws. There is an embedding of $S^3$ into $\H$ since $\H$ is diffeomorphic to $\R^4$ in the obvious way. Furthermore, one observes that for $x_1,x_2 \in S^3$ with $s_l = (a_l,b_l,c_l,d_l)$ in $\H$, one has that
\begin{align*}
  s_1s_2 = &(a_1a_2 -b_1b_2 -c_1c_2 -d_1d_2)^2 + (a_1b_2 + a_2b_1 + c_1d_2 - c_2d_1)^2 \\
          &+(a_1c_2 + a_2c_1 -b_1d_2 + b_2d_1)^2 + (a_1d_2 + a_2d_1 + b_1c_2 - b_2c_1)^2 \\
\end{align*}
The final expression equals to $(a_1^2 + b_1^2 + c_1^2 + d_1^2)(a_2^2+b_2^2+c_2^2+d_2^2) = 1$. Therefore multiplication of quarternions restricts to a multiplication on $S^3$, and so does inversion: $\H$ is a skew field, so non-zero elements have inverses. If $s$ has an inverse $s'$ then $|s|\cdot |s'| = 1$. \\

Hence $S^3$ is a Lie group. Any Lie group has trivial tangent bundle, since a basis of the Lie algebra can be translated by group elements to bases of other tangent spaces yielding linearly independent sections of maximum rank.

\section*{Problem 3}
Suppose $v_1, \dots, v_k$ are linearly independent. This automatically means $k \leq n$. Extend these vectors to a basis $\mathcal B$ of $V$. Consider the map $\det_m: V^k \to \R$ which takes any tuple of vectors $(w_1, \dots, w_k)$ to the determinant of the first $k\times k$ minor (recall, $k\leq n$) of the $n\times k$ matrix of $(w_1,\dots,w_k)$ written in the basis $\mathcal B$. Clearly, this map is multilinear and alternating in the $k$ arguments, since taking determinant is multilinear and alternating in the columns. It follows that $\det_m$ factors through $\Lambda^k V$. Also observe that $\det_m(v_1,\dots, v_k) = 1$ by definition, so $v_1, \dots, v_k$ cannot map to $0$ in $\Lambda^k V$. This means if $v_1, \dots, v_k$ are linearly independent then $v_1 \wedge \dots \wedge v_k$.\\

If $v_1, \dots, v_k$ are linearly dependent, then write one of the $v_i$ as a linear combination of the other $v_j$'s and use multilinearity to expand the wedge $v_1\wedge \dots \wedge v_k$ at the $i$-th place. In this expansion, each wedge has two same entries and is hence zero. Therefore, $v_1 \wedge \dots \wedge v_k = 0$. 

\section*{Problem 4}
Let $A = A_{ij}$ in the bases $v_1, \dots, v_n$ and $w_1, \dots, w_n$. Then $Av_i = \sum_j A_{ij}w_j$. Using multilinearity at each coordinate and removing terms which have two same $w_j$'s, one has that

\begin{align*}
  Av_1 \wedge \dots \wedge Av_n &= \bigwedge_{i=1}^n\left (\sum_j A_{ij}w_j\right )\\
                                &= \sum_\sigma \bigwedge_{i=1}^n A_{i\sigma(i)}w_{\sigma(i)} \\
                                &= \sum_\sigma \left (\mathrm{sgn}(\sigma)\prod_i A_{i\sigma(i)} w_1 \wedge \dots \wedge w_n \right)\\
                                &= \det (A)\cdot w_1\wedge \dots \wedge w_n.
\end{align*}

\section*{Problem 5}
(a) We will show that the first form does not extend to all of $\C\P^1$. The argument for the other form is similar.

Consider the change of coordinates between the charts: \[(x,y) \mapsto \left( \frac x{x^2 + y^2}, \frac{-y}{x^2+y^2} \right ).\] Therefore, in the second chart $(u,v)$, the same form must be the form (after some simple calculations) \[ \gamma = \frac {u^2 - v^2}{(1+u^2+v^2)(u^2+v^2)}du + \frac{-2uv}{(1+u^2+v^2)(u^2+v^2)}dv.\] However, this form does not extend (continuously) to the point $(u,v) = (0,0)$ since different directions towards $(0,0)$ give different values of the limit of $\gamma$.\\

(b) Suppose otherwise that there are $1$-forms $\alpha, \beta$ for which $\omega = \alpha \wedge \beta$. Since $\omega$ is non-vanishing, $\alpha, \beta$ must be linearly independent on each fiber. Choosing dual bases of $\alpha, \beta$ at each tangent space, one obtains vector fields $\alpha^*, \beta^*$ which are linearly independent on each fiber. In particular, that means $\alpha^*$ is a non-vanishing vector field on $S^2$ which does not exist for Euler characteristic reasons.

\section*{Problem 6}
We claim that the differential form in complex coordinates $(w_1,w_2)$ of $U_1$ is given by
\begin{align*}                                                                                                                                                                                                                                              
\omega = \dfrac{i}{2} \left( \dfrac{1}{(1 + |w_1|^2 + |w_2|^2)^2} \right) &\Big((1 + |w_2|^2)dw_1 \wedge d\bar{w}_1 + (1 + |w_1|^2)dw_2 \wedge d\bar{w}_2 \\                                                                                                           
&- \bar{w}_1w_2 dw_1 \wedge \bar{w}_2 - \bar{w}_2 w_1 dw_2 \wedge d\bar{w}_1)\Big).                                                                                                                                                                           
\end{align*}

We will only compute the coefficients of wedges $dw_1 \wedge d\bar w_1$ and $dw_1\wedge d\bar w_2$. The other two computations are similar. One has that $z_1 = \frac 1{w_1}$, $z_2 = \frac{w_2}{w_1}$. Under this change of coordinates, we have
\begin{align*}
  dz_1 & = -\frac1{w_1^2} dw_1,\\
  dz_2 &= -\frac{w_2}{w_1^2} dw_1 + \frac1{w_1}dw_2,\\
  d\bar z_1 &= \frac1{\bar w_1^2} d\bar w_1,\\
  d\bar z_2 &= -\frac{\bar w_2}{\bar w_1^2} d\bar w_1 - \frac1{\bar w_1}d\bar w_2.
\end{align*}
Using these expressions, we painfully compute the coefficient of $dw_1 \wedge d\bar w_1$ in $\omega$ to be \[\frac i2 \frac{1}{(1+|w_1|^2+|w_2|^2)^2}\left(1 + \frac{|w_2|^2}{|w_1|^2} + |w_2|^2 + \frac{|w_2|^2}{|w_1|^2} - 2 \frac{|w_2|^2}{|w_1|^2}\right) = \frac i2 \frac{1}{(1+|w_1|^2+|w_2|^2)^2} \left(1 + |w_2|^2\right).\]
Also, the coefficient of $dw_1 \wedge d\bar w_2$ is computed to be \[\frac i2 \frac{1}{(1+|w_1|^2+|w_2|^2)^2}\left((-w_2\bar w_1)\left(1+\frac{1}{|w_1^2|}\right) - \frac{w_2}{|w_1^2|} (-\bar w_1)\right) = \frac i2 \frac{1}{(1+|w_1|^2+|w_2|^2)^2}(-\bar w_1w_2).\]


\end{document}
