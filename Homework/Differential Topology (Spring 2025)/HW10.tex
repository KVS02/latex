\documentclass{amsart}

\usepackage{amsmath,amssymb,amsfonts,amscd}
\usepackage[all,cmtip]{xy}
\usepackage{enumerate}
\usepackage{ amssymb, latexsym, amsmath}
%\setcounter{MaxMatrixCols}{30}%
\usepackage{fullpage}
\usepackage{url}
\usepackage{hyperref}
\usepackage{ marvosym }
\usepackage{quiver}



\numberwithin{equation}{section}

%\usepackage{cjw-latex}

%\providecommand{\U}[1]{\protect\rule{.1in}{.1in}}

\theoremstyle{plain}
\newtheorem{theorem}{Theorem}[section]
\newtheorem*{thm}{Theorem}
\newtheorem{proposition}[theorem]{Proposition}
\newtheorem{lemma}[theorem]{Lemma}
\newtheorem{conjecture}[theorem]{Conjecture}
\newtheorem{corollary}[theorem]{Corollary}
\newtheorem{algorithm}[theorem]{Algorithm}
\newtheorem{axiom}[theorem]{Axiom}
\newtheorem{criterion}[theorem]{Criterion}
%\newtheorem{conjecture}[theorem]{Conjecture}
\newtheorem{wildconjecture}[theorem]{Wild Conjecture}


\theoremstyle{definition}
\newtheorem{definition}[theorem]{Definition}
\newtheorem*{dfn}{Definition}
\newtheorem{condition}[theorem]{Condition}
\newtheorem{example}[theorem]{Example}
\newtheorem{exercise}[theorem]{Exercise}
\newtheorem{notation}[theorem]{Notation}
\newtheorem{question}[theorem]{Question}
\newtheorem{problem}[theorem]{Problem}
\newtheorem{solution}[theorem]{Solution}




\theoremstyle{remark}
\newtheorem{remark}[theorem]{Remark}
\newtheorem{remarks}[theorem]{Remarks}
\newtheorem{summary}[theorem]{Summary}
\newtheorem{observation}[theorem]{Observation}
\newtheorem{conclusion}[theorem]{Conclusion}
\newtheorem{acknowledgement}[theorem]{Acknowledgement}
\newtheorem{case}[theorem]{Case}
\newtheorem{claim}[theorem]{Claim}



\makeatletter
\newcommand{\rmnum}[1]{\romannumeral #1}
\newcommand{\Rmnum}[1]{\expandafter\@slowromancap\romannumeral #1@}
\makeatother
\newcommand{\Aut}{\operatorname{Aut}}
\newcommand{\Ext}{\operatorname{Ext}}
\newcommand{\Tor}{\operatorname{Tor}}
\newcommand{\Z}{\mathbb{Z}}
\newcommand{\osum}{\oplus}
\newcommand{\aff}{\operatorname{Aff}}

\newcommand{\Id}{\operatorname{Id}}
\newcommand{\concat}{(\Id \to \Omega \Sigma)}
\newcommand{\eval}{(\Sigma\Omega \to \Id)}
\newcommand{\Q}{\mathbb{Q}}
\newcommand{\F}{\mathbb{F}}
\newcommand{\N}{\mathbb{N}}
\newcommand{\g}{\mathfrak{g}}
\newcommand{\n}{\mathfrak{n}}
\newcommand{\h}{\mathfrak{h}}
\newcommand{\p}{\mathfrak{p}}
\newcommand{\q}{\mathfrak{q}}
\newcommand{\m}{\mathfrak{m}}
\newcommand{\e}{\mathfrak{e}}
\newcommand{\f}{\mathfrak{f}}
\newcommand{\V}{\mathscr{V}}
\newcommand{\D}{\mathscr{D}}
\renewcommand{\S}{\mathscr{S}}
\newcommand{\Sch}{\mathscr{C}}
\newcommand{\Y}{\mathscr{Y}}
\renewcommand{\u}{\mathfrak{u}}
\newcommand{\set}[2]{ \left\{ {#1} \, \left| \, {#2} \right\}\right.}
\newcommand{\A}{\mathbb{A}}
\newcommand{\X}{\mathcal{X}}
\renewcommand{\Im}{\operatorname{Im}}
\newcommand{\T}{\mathscr{T}}
\newcommand{\HH}{\mathscr{H}}
\newcommand{\PGL}{\operatorname{PGL}}
\newcommand{\righthookarrow}{\hookrightarrow}
\newcommand{\sign}{\operatorname{sign}}
\renewcommand{\_}[2]{\underbrace{#1}_{#2}}
\renewcommand{\^}[2]{\overbrace{#1}_{#2}}
\newcommand{\curverightarrow}{\curvearrowright}

\newcommand{\z}{\mathfrak{z}}
\renewcommand{\sl}{\mathfrak{sl}}
\newcommand{\R}{\mathbb{R}}
\newcommand{\RP}{\mathbb{RP}}
\renewcommand{\P}{\mathbb{P}}
\newcommand{\C}{\mathbb{C}}
\renewcommand{\H}{\mathcal{H}}
\newcommand{\Ind}{\operatorname{Ind}}
\newcommand{\Tr}{\operatorname{Tr}}
\newcommand{\Orb}{\mathcal{O}}
\renewcommand{\k}{\mbox{\Fontauri k}}
\newcommand{\Ad}{\operatorname{Ad}}
\newcommand{\ad}{\operatorname{ad}}
\newcommand{\Hom}{\operatorname{Hom}}
\newcommand{\Ker}{\operatorname{Ker}}
\newcommand{\End}{\operatorname{End}}
\newcommand{\supp}{\operatorname{supp}}
\newcommand{\Stab}{\mbox{Stab}}
\newcommand{\Lie}{\operatorname{Lie}}
\newcommand{\GL}{\operatorname{GL}}
\newcommand{\SL}{\operatorname{SL}}
\newcommand{\Sp}{\operatorname{Sp}}
\newcommand{\Mp}{\operatorname{Mp}}
\renewcommand{\a}{\mathfrak{a}}
\newcommand{\Wh}{\operatorname{Wh}}
\newcommand{\Span}{\operatorname{Span}}
\newcommand{\WF}{\operatorname{WF}}
\newcommand{\AC}{\operatorname{AC}}
\newcommand{\Lin}{\operatorname{Lin}}
\newcommand{\Diff}{\operatorname{Diff}}
\newcommand{\Gen}{\operatorname{Gen\, Hom}}
\newcommand{\Her}{\operatorname{Her}}
\newcommand{\pivot}{&}
\newcommand{\sgn}{\operatorname{sgn}}
%\newcommand{C^{\infty}}{DS}
\newcommand{\vsp}{{\vspace{0.2in}}}



\title{Homework 10}

\author{Sudharshan K V}
\begin{document}
\maketitle

\section*{Problem 1}
(a) Let $\omega$ be a form. Let $U_i$, $\rho_i$ be a cover (with standard orientation) and an associated partition of unity. Let $V_j, \psi_j$ be another cover, partition of unity. We use without proof that in $\R^n$, the integration of a function is the sum over integrating the function over a partition of unity. Then, one has
\begin{align*}
  \sum_i \int_{U_i}\rho_i\omega &= \sum_i \int_{U_i} \sum_j\psi_j\rho_i\omega\\
                                &= \sum_i \sum_j \int_{U_i\cap V_j} \rho_i\psi_j \omega.\\
                                &= \sum_{i,j} \int_{U_i\cap V_j} \rho_i\psi_j\omega.
\end{align*}
The final expression is symmetric in $i,j$ so the integral of $\omega$ defined wrt $U_i, \rho_i$ is equal to that with respect to $V_j, \psi_j$. 
\\

(b) For $0$-dim manifolds, this is clear. Assume dimension of $X$ is positive. First observe that for a top form $f(x) dx_1\wedge \dots \wedge dx_n$ on $\R^n$, changing the orientation of $\R^n$ means that the form on $\R^n$ in the chart $x_1 \mapsto -x_1$ is just $f(-x_1, \dots, x_n) dx_1 \wedge \dots \wedge dx_n$. Therefore, in charts, one has $\int_U \omega$ = $-\int_{-U}\omega$. Extending to all of $X$ follows from definition.

\section*{Problem 2}
For dimension $0$ this is clear. So we do for positive dimension $X$. Suppose there is a non-vanishing top form $\omega$ on $X$. For a chart $U$ on $X$, $\omega$ in coordinates $U$ is of the form $f(x) dx_1 \wedge \dots \wedge dx_n$, where $f(x)$ is always positive or always negative. Modify the chart so that $f(x)$ is always positive. Consider the collection of all such charts. Now the transition map between any two charts has positive determinant, for in both charts in the standard coordinates the form $\omega$ is positive and it changes by the determinant under the change of coordinates. This shows $X$ is orientable.\\

Conversely, suppose that $X$ is orientable, so choose a cover $U_i$ of $X$ so that all transition maps have positive determinant. Choose a partition of unity $\rho_i$ subordinate to $U_i$, and consider the forms $\omega_i = \rho_i dx_1 \wedge \dots \wedge dx_n$. One can extend these forms to all of $X$ by setting $\omega_i$ to be $0$ outside $U_i$. Thus we have a set of globally defined forms $\omega_i$. Consider $\omega = \sum_i \omega_i$. This is a finite sum at each point due to para-compactness. Furthermore, at any point, choosing a chart with standard orientation, all the $\omega_i$ have non-negative coefficient. Also, $\sum rho_i = 1$ so $\omega$ has a positive coefficient. This shows $\omega$ is non-vanishing top form on $X$.

\section*{Problem 3}
Under the coordinates $(0,2\pi) \to S^1$, $\theta \mapsto (\cos\theta, \sin \theta)$, one computes that the form given is just $\sin^2\theta d\theta + \cos^2 \theta d\theta = d\theta$. The integral of this form on $S^1$ is $2\pi$. (The chart misses one point which does not contribute to the integral.) See Problem 9(c) where a similar form gives $d\theta$ on circles.

\section*{Problem 4}

Let $\omega$ be the two form we integrate over $S^2$. We compute $d\omega = 3 dx\wedge dy \wedge dz$. Let $B$ be the closed unit ball in $\R^3$. Giving $S^2$ the standard orientation, we see that $S^2 = \partial B$. By Stokes' theorem, \[\int_B d\omega = \int_{S^2}\omega.\] The first integral is just thrice the volume of the unit sphere, which is $3\times \frac 43 \pi = 4\pi$.

\section*{Problem 5}
The only form which pulls back to something non-trivial on $\C\P^1$ is $dz_1 \wedge d\bar z_1$, and on the standard copy, $z_2= 0$. Removing a point $[0:1:0]$ from this allows us to change coordinates to $\R^2$ with $[1:x+iy:0] \mapsto (x,y)$. This preserves orientation by definition of orientation on $\C\P^1$. The integral of this form is \[\int_{\R^2} \frac{1}{(1+x^2+y^2)^2} dx\wedge dy = \int_0^{2\pi}\int_0^\infty \frac{r}{(1+r^2)^2} drd\theta.\] The final integral evaluates to $2\pi$ by M 408D.

\section*{Problem 6}

Consider the function $\ell: V^k \to \R$ given by \[(v_1 \dots v_k) \mapsto \sum_\sigma \sgn(\sigma) \prod_i (\alpha_i(v_{\sigma(i)})).\] This is clearly multilinear on the coordinates $v_i$, and is alternating too since swapping two of the $v_i$ translates the sum over $\sigma$ by a transposition, which contributes a $-1$ to the sign. Hence this map descends to $\wedge^k V \to \R$. Similarly, one observes that these coordinates are multilinear and alternating in the $\alpha_i$ too, so the map $\eta$ is well-defined.\\

We will show that the map $\eta$ is injective. Due to the spaces being equidimensional, $\eta$ becomes an isomorphism. Choose $\alpha_1 \wedge \dots \wedge \alpha_k \neq 0$ in $\wedge^kV^*$. This means $\alpha_i$ are linearly independent. Extend the $\alpha_i$ to a basis of $V^*$, and let $v_i$ be the dual basis. Then evaluating the map $\ell = \eta(\alpha_i)$ on $v_1, \dots , v_k$ gives us $1$ since the only permutation which contributes is the identity. So $\ell$ is not the $0$ map. Therefore $\eta$ is injective and the isomorphism follows.\\

WLOG assume that in the coordinates of $U$, $Z \subset X$ is the first $k$-coordinates inclusion $\R^k \subset \R^n$. Writing $\omega$ as a linear combination of standard $k$-forms, the only form which non-trivially pulls back to $Z$ is the form $g(x_1, \dots, x_k, 0, 0,...) dx_1 \wedge \dots \wedge dx_k$. Also note that $f(p) = \eta(\omega|_p) (\partial/\partial x_1 \wedge \dots \partial/\partial x_k)$ since there are no other pullbacks. But this just evaluates to \[f(p) = g(p,0,..,0),\] since the only  permutation which gives a non-zero value is the identity permutation, and that evaluates to $1$. Now the equality of integrals is by definition. 

\section*{Problem 7}
(a) For $0$-forms, this statement is obvious. Take a $1$-form $\omega = g(y) dy_1$ in coordinates $y_1$. In the change of coordinates under the map $f: x_1, \dots, x_n \mapsto y_1, \dots, y_n$, this form is given by \[\omega = g(f(x))\sum_i f_i(x) dx_i,\] where $f_i$ is the partial of $f$ wrt $x_i$. Computing $d\omega$ in the first coordinate system, one has $d\omega = \sum_j g_j dy_j \wedge dy_1$. In the second coordinate system, one has \[d\omega = \sum_i \sum_j \partial/\partial x_j (g(f(x)) f_i(x) dx_j\wedge dx_i.\] Swapping the summations, one finds that this form is exactly the form $d\omega$ in coordinates $y$ written in the coordinates $x$ due to the chain rule. Therefore, on $1$-forms, $d$ is well-defined on the coordinates. Also note that the definition of $d$ for higher forms is determined by $d$ defined on the lower forms since in (some) coordinates, $d(\omega_1 \wedge \omega_2) = d\omega_1 \wedge \omega_2 + (-1)^k \omega_1 \wedge d\omega_2$. Since $d$ is coordinate independent on $1$-forms, we see that $d$ is coordinate independent on all higher forms.\\

(b) For $0$-forms, this statement is obvious. Indeed, for a function $g$, we have $df^*(g)$ = $d(f(g)) = \sigma (f(g))_i(x) dx_i$ and this is equal by chain rule and linearity of pullback to $f^*(dg)$. We follow a similar approach to part (a). In fact, in (a) we just proved that the pullback along a transition map (change of coordinates) commutes with $d$. Indeed we never used the fact that $f$ was a diffeomorphism anywhere; we only used that it was a differentiable map. So for $1$-forms $\omega$, we thus have $f^*(d\omega) = df^*(\omega)$. However, for higher forms, we just need to note that $f^*(\omega_1 \wedge \omega_2) = f^*(\omega_1) \wedge f^*(\omega_2).$ This follows from the abstract definition of $f^*$, since the natural maps $\bigwedge^k V \times \bigwedge^l V \to \bigwedge^{k+l}V$ for $V = T^*_xX,T^*_{f(x)}Y$ commute with the maps induced by $f$ on the exterior power of the cotangent spaces. Due to a similar argument, we find that for forms $\omega$ of all order, $f^*d\omega = df^*\omega$.
\section*{Problem 8}
The pullback commutes with the exterior derivative, so one has $\gamma^*(df) = d(\gamma^*f)$. But $\gamma^*f = f\circ \gamma$ since $f$ is a $0$-form. Thus, \[\int_a^b \gamma^*(df) = \int_a^b d(f\gamma) = f(\gamma(b)) - f(\gamma(a)) = f(q)-f(p),\] by the fundamental theorem of calculus. 

\section*{Problem 9}
(a) A simple calculation yields \[d\omega = \frac{1}{(x^2+y^2)^2} \left( (y^2-x^2) dy \wedge dx + (y^2 - x^2)dx \wedge dy\right) = 0.\]\\

(b) Consider the function $f(x,y) = \arctan (y/x)$ ranging in $(-\tfrac\pi2, \tfrac\pi2)$. This is a well-defined smooth function on the right half-plane. One computes that \[df = \frac{1}{x^2+y^2} \left(-y dx + x dy \right).\]

(c) The integral of $\omega$ over $S^1$ is $2\pi$. (See that the pullback of $\omega$ via the inclusion map $S^1 \to \R^2-0$ is just the form $d\theta$; this also motivates the selection of the function $f$ in part (b), which returns $\theta$ given $(x,y)$.) It follows that $\omega$ is not exact.

\end{document}
