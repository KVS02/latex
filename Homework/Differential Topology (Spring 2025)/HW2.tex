\documentclass{amsart}

\usepackage{amsmath,amssymb,amsfonts,amscd}
\usepackage[all,cmtip]{xy}
\usepackage{enumerate}
\usepackage{ amssymb, latexsym, amsmath}
%\setcounter{MaxMatrixCols}{30}%
\usepackage{fullpage}
\usepackage{url}
\usepackage{hyperref}
\usepackage{ marvosym }




\numberwithin{equation}{section}

%\usepackage{cjw-latex}

%\providecommand{\U}[1]{\protect\rule{.1in}{.1in}}

\theoremstyle{plain}
\newtheorem{theorem}{Theorem}[section]
\newtheorem*{thm}{Theorem}
\newtheorem{proposition}[theorem]{Proposition}
\newtheorem{lemma}[theorem]{Lemma}
\newtheorem{conjecture}[theorem]{Conjecture}
\newtheorem{corollary}[theorem]{Corollary}
\newtheorem{algorithm}[theorem]{Algorithm}
\newtheorem{axiom}[theorem]{Axiom}
\newtheorem{criterion}[theorem]{Criterion}
%\newtheorem{conjecture}[theorem]{Conjecture}
\newtheorem{wildconjecture}[theorem]{Wild Conjecture}


\theoremstyle{definition}
\newtheorem{definition}[theorem]{Definition}
\newtheorem*{dfn}{Definition}
\newtheorem{condition}[theorem]{Condition}
\newtheorem{example}[theorem]{Example}
\newtheorem{exercise}[theorem]{Exercise}
\newtheorem{notation}[theorem]{Notation}
\newtheorem{question}[theorem]{Question}
\newtheorem{problem}[theorem]{Problem}
\newtheorem{solution}[theorem]{Solution}




\theoremstyle{remark}
\newtheorem{remark}[theorem]{Remark}
\newtheorem{remarks}[theorem]{Remarks}
\newtheorem{summary}[theorem]{Summary}
\newtheorem{observation}[theorem]{Observation}
\newtheorem{conclusion}[theorem]{Conclusion}
\newtheorem{acknowledgement}[theorem]{Acknowledgement}
\newtheorem{case}[theorem]{Case}
\newtheorem{claim}[theorem]{Claim}



\makeatletter
\newcommand{\rmnum}[1]{\romannumeral #1}
\newcommand{\Rmnum}[1]{\expandafter\@slowromancap\romannumeral #1@}
\makeatother
\newcommand{\Aut}{\operatorname{Aut}}
\newcommand{\Ext}{\operatorname{Ext}}
\newcommand{\Tor}{\operatorname{Tor}}
\newcommand{\Z}{\mathbb{Z}}
\newcommand{\osum}{\oplus}
\newcommand{\aff}{\operatorname{Aff}}

\newcommand{\id}{\operatorname{Id}}
\newcommand{\concat}{(\Id \to \Omega \Sigma)}
\newcommand{\eval}{(\Sigma\Omega \to \Id)}
\newcommand{\Q}{\mathbb{Q}}
\newcommand{\F}{\mathbb{F}}
\newcommand{\N}{\mathbb{N}}
\newcommand{\g}{\mathfrak{g}}
\newcommand{\n}{\mathfrak{n}}
\newcommand{\h}{\mathfrak{h}}
\newcommand{\p}{\mathfrak{p}}
\newcommand{\q}{\mathfrak{q}}
\newcommand{\m}{\mathfrak{m}}
\newcommand{\e}{\mathfrak{e}}
\newcommand{\f}{\mathfrak{f}}
\newcommand{\V}{\mathscr{V}}
\newcommand{\D}{\mathscr{D}}
\renewcommand{\S}{\mathscr{S}}
\newcommand{\Sch}{\mathscr{C}}
\newcommand{\Y}{\mathscr{Y}}
\renewcommand{\u}{\mathfrak{u}}
\newcommand{\set}[2]{ \left\{ {#1} \, \left| \, {#2} \right\}\right.}
\newcommand{\A}{\mathcal{A}}
\newcommand{\X}{\mathcal{X}}
\newcommand{\B}{\mathcal{B}}
\renewcommand{\Im}{\operatorname{Im}}
\newcommand{\T}{\mathscr{T}}
\newcommand{\HH}{\mathscr{H}}
\newcommand{\PGL}{\operatorname{PGL}}
\newcommand{\righthookarrow}{\hookrightarrow}
\newcommand{\sign}{\operatorname{sign}}
\renewcommand{\_}[2]{\underbrace{#1}_{#2}}
\renewcommand{\^}[2]{\overbrace{#1}_{#2}}
\newcommand{\curverightarrow}{\curvearrowright}

\newcommand{\z}{\mathfrak{z}}
\renewcommand{\sl}{\mathfrak{sl}}
\newcommand{\R}{\mathbb{R}}
\newcommand{\RP}{\mathbb{RP}}
\renewcommand{\P}{\mathbb{P}}
\newcommand{\C}{\mathbb{C}}
\renewcommand{\H}{\mathcal{H}}
\newcommand{\Ind}{\operatorname{Ind}}
\newcommand{\Tr}{\operatorname{Tr}}
\newcommand{\Orb}{\mathcal{O}}
\renewcommand{\k}{\mbox{\Fontauri k}}
\newcommand{\Ad}{\operatorname{Ad}}
\newcommand{\ad}{\operatorname{ad}}
\newcommand{\Hom}{\operatorname{Hom}}
\newcommand{\Ker}{\operatorname{Ker}}
\newcommand{\End}{\operatorname{End}}
\newcommand{\supp}{\operatorname{supp}}
\newcommand{\Stab}{\mbox{Stab}}
\newcommand{\Lie}{\operatorname{Lie}}
\newcommand{\GL}{\operatorname{GL}}
\newcommand{\SL}{\operatorname{SL}}
\newcommand{\Sp}{\operatorname{Sp}}
\newcommand{\Mp}{\operatorname{Mp}}
\renewcommand{\a}{\mathfrak{a}}
\newcommand{\Wh}{\operatorname{Wh}}
\newcommand{\Span}{\operatorname{Span}}
\newcommand{\WF}{\operatorname{WF}}
\newcommand{\AC}{\operatorname{AC}}
\newcommand{\Lin}{\operatorname{Lin}}
\newcommand{\Diff}{\operatorname{Diff}}
\newcommand{\Gen}{\operatorname{Gen\, Hom}}
\newcommand{\Her}{\operatorname{Her}}
\newcommand{\pivot}{&}
\newcommand{\sgn}{\operatorname{sgn}}
%\newcommand{C^{\infty}}{DS}
\newcommand{\vsp}{{\vspace{0.2in}}}
\newcommand{\Gr}{\operatorname{Gr}}


\title{Homework 2}

\author{Sudharshan K V}
\begin{document}
\maketitle

\section*{Problem 1}
(a) Let $\A_1$ and $\A_2$ be (not necessarily maximal) smooth atlases on $X$. Recall that the maximal atlas $\hat \A$ generated by an atlas $\A$ is the set of smooth maps on open sets of $X$ which are (smoothly) compatible with all the charts of $\A$. If all charts of $\A_2$ are compatible with all charts of $\A_1$, then $\A_2 \subset \hat \A_1$. Since $\hat \A$ is a maximal atlas, the maximal atlases generated by $\A_1, \A_2$ coincide. (It is clear that each chart of $\A_1$ being compatible with every chart of $\A_2$ is the same as every chart of $\A_2$ being compatible with all charts of $\A_1$.)

Conversely, if $\hat \A_1 = \hat \A_2$, then clearly $\A_2 \subset \hat \A_1$. Consequently, every chart of $\A_2$ is compatible with all charts of $\A_1$.\\

(b) Here, $\A_i$ are maximal atlases. Let $(U, \varphi_U: U \to \R^n)$ be a chart in $\A_2$ which is not in $\A_1$. Since $\A_1$ is maximal, there is some chart $(V, \varphi_V: V \to \R^n)$ in $\A_1$ which is not compatible with $\varphi_U$. Choose the coordinate $i$ so that the $i$-th coordinate of the transition function from $\varphi_V(V)$ to $\varphi_U(U)$ is not differentiable, and let $f = \varphi_{U,i}$. Then $f$ is smooth wrt $\A_2$. However, $f$ is not smooth wrt $\A_1$ because the map $$f \circ \varphi_V^{-1} =  (\varphi_U \circ \varphi_V^{-1})_i: \varphi_V(U\cap V) \to \R$$ is not differentiable (by the choice of $i$). $f$ is therefore a map smooth wrt one atlas but not the other. 

\section*{Problem 2}

(a) If the atlases $\A_1$ and $\A_2$ are the same, then the identity map is a diffeomorphism: when written in coordinates, the map of Euclidean spaces is the identity (both directions), which is smooth.

Suppose that the identity map $\id: (X,\A_1)\to (X, \A_2)$ is a diffeomorphism. If $\A_1 \neq \A_2$, then by Problem 1(b), there is some function $f:X \to Y(f: X \to \R^m)$ which is smooth wrt $\A_2$ but not wrt $\A_1$. However, composition of smooth maps is smooth, and $\id: (X,\A_1) \to (X,\A_2)$ being smooth means that $f$ is smooth wrt $\A_1$ as well. This is a contradiction, so $\A_1$ must be the same as $\A_2$.\\

(b) Let $f: \R \to \R$ be any homeomorphism. (For instance, $f(x) = x^n$ for odd $n$.) Let $X = R$ as a topological space (so when we say $\R$, we mean the usual real line) and consider $f$ as a chart from $X \to \R$ . Endow $X$ with a smooth structure, using the maximal atlas generated by $f$. We will denote $X$ as a manifold with this structure by $(X,f)$. If $f: x\mapsto x^n$ for odd $n>0$, we denote the manifold structure by $X_n$.

See that $(X,f)$ is diffeomorphic to $\R$ with the standard atlas through the map $f: (X,f) \to \R$. In coordinates this map is just the identity function on $\R$, which is a diffeomorphism. So all the $(X,f)$ are diffeomorphic. Next, we show that the atlases are different.

The identity map $\id: (X,f) \to (X,g)$ is smooth iff $gf^{-1}: \R \to \R$ is smooth. It is a diffeomorphism iff both $gf^{-1}$ and $fg^{-1}$ are smooth. For $m,n$ distinct, odd and positive, the maps $x \mapsto x^{\frac m n}$ is not a smooth map on $\R$ when $n > m$. Therefore, the identity map $X_n \to X_m$ is not a diffeomorphism unless $m = n$. Therefore, the smooth structures $X_n$ on $X$ for odd positive integers $n$ have different atlases.

\section*{Problem 3}
First we note that the map $f: [x:y] \mapsto [x^2: xy : y^2]$ is well-defined. Trivially, all of $x^2,y^2,xy$ cannot be zero unless both $x,y$ are zero. If one has $(x,y) = \lambda (x',y')$ for some non-zero $\lambda$, then $f([x:y]) = \lambda^2 f([x':y'])$, so equivalence classes under scalar multiplication are mapped to the same element of $\R\P^2$. 

Let $P = [x:y]$ be a point in $\R\P^1$. We show that $f$ is smooth at $p$. WLOG (symmetry), assume that $x \neq 0$, and write $P = [1:p]$. Then $f(P) = [1:p:p^2]$. Consider the chart on $\R\P^1$ which sends $[1:y]$ to $y$, and the chart on $\R\P^2$ which sends $[1:y:z] \to (y,z)$. The smoothness of $f$ at $P$ is the same as the composite $p \mapsto [1:p] \stackrel{f}{\mapsto} [1:p:p^2] \mapsto (p,p^2)$ being smooth, and that is clear. 

\section*{Problem 4}
Let $n, s$ be the north and south poles of $S^1$. Let $f: S^1 - n \to \R$ be the stereographic projection from $n$, and let $g: S^1 - n \to \R$ be the stereographic projection from $s$. Recall that the transition map from between $f$ and $g$ is given by $x \mapsto 4/x$. Scale $g$ by $4$ (it is still a chart) so that the transition map is now $x \mapsto 1/x$.

Define the map $\Phi: S^1 \to \R\P^1$ as follows.
\begin{align*}
  \Phi: & x \mapsto  [f(x):1] & x \neq n,\\
  : & n \mapsto  [0:1]
\end{align*}
Then, clearly $\Phi$ is smooth by definition at all points $x$ different from $n$ - the map $\Phi$ in the obvious chart coordinates is the identity map. It remains to check smoothness at $n$. Consider the following alternate description of $\Phi$.
\begin{align*}
  \Phi: & x \mapsto  [1:g(x)] & x \neq s, \\
  : & s \mapsto  [0:1]
\end{align*}
For $x$ different from $n,s$, these two definitions for $\Phi(x)$ agree because $f(x) g(x) = 1$. One can check that they agree on $n$ and $s$ as well. Using the second description, it is easy to see that $\Phi$ is smooth at $n$: $\Phi$ in coordinates is again the identity map. So $\Phi$ is smooth.

It is also easy to see that $\Phi$ is a bijection. Since $\Phi$ is the identity map in coordinates, so is $\Phi^{-1}$. Hence, $\Phi$ is a diffeomorphism of $S^1$ to $\R\P^1$.

\section*{Problem 5}
If $f$ is a constant polynomial $c$, extend $f$ to $\R\P^1$ by mapping everything to $[c:1]$. Constant maps are smooth. From now, suppose that $f$ has degree at least $1$.

Define the map $\tilde f: \R\P^1 \to \R\P^1$ by $[x:1] \mapsto [f(x):1]$ for and $[1:0] \mapsto [1:0]$. The smoothness of $\tilde f$ at points different from $[1:0]$ is clear: written in coordinates, it is just the map $f$, which is smooth. For $\varepsilon \neq 0$, $$\tilde f([1:\varepsilon])= [f(1/\varepsilon):1] = [1:f(1/\varepsilon)^{-1}],$$ where $\varepsilon$ is chosen small enough that $f$ does not have zeroes beyond $1/\varepsilon$. Therefore, in the chart given by $[1:x] \mapsto x$, the map $\tilde f$ looks like $x \mapsto f(1/x)^{-1}$ near $0$. And $\tilde f$ takes $0\to 0$, so we need to show $f(1/x)^{-1}$ is differentiable at $0$. This is clear by writing down $f(x) = \sum_0^n a_ix^i$ with $a_n \neq 0$, and we have $$f(1/x)^{-1} = \frac{x^n}{a_n(1 + b_1x + b_2 x^2 + \dots + b_{n}x^{n})},$$ which is differentiable around $0$ since the denominator is non-zero.

\section*{Problem 6}
We produce charts on $\Gr(n,k)$ as follows. Let $\Delta = \{ e_1, \dots, e_n\}$ be the standard basis of $\R^n$. Let $S$ be a subset of $\Delta$, and let $V_S$ denote the span of $S$.

For each $S \subset \Delta$ with size $n-k$, consider the set \[\Gr^S(n,k) = \{ V \in \Gr(n,k) : V \cap V_S = 0\}.\]
The condition $V\cap V_S$ is equivalent to $V + V_S = \R^n$, by comparing dimensions. We provide charts for $\Gr^S(n,k)$ for each $S$. Indeed, every such $V$ in $\Gr^S(n,k)$ is a graph of a linear function $V_{\Delta \setminus S} \to V_S$: There is a canonical map $V_S \times V_{\Delta \setminus S} \simeq \R^n$. Realizing $V$ as a subspace of this product along with $V \cap V_S = 0$ makes $V$ a graph of a (linear) function $f_V: V_{\Delta \setminus S} \to V_S$. Conversely, given a linear function $f: V_{\Delta \setminus S} \to V_S$, the graph of $f$ is a linear subspace of $\R^n$ with dimension $|\Delta \setminus S| = k$. The association $\varphi_S: V \to f_V$ is a chart from $\Gr^S(n,k) \to R^{k \times (n-k)}$, once we identify maps from $V_{\Delta \setminus S} \to V_S$ with $k\times n-k$ matrices using the standard bases for both the domain and codomain. \\

For $n=4, k=2$, we will compute one transition map. We choose $S_1 =\{1,2\}$ and $S_2 = \{3,4\}$. Let $\Gr^i = \Gr^{S_i}(4,2)$ and $V_i = V_{S_i}$. For $V \in \Gr^1 \cap \Gr^2$, we see that $V$ is a graph of a linear function from $V_1 \to V_2$ and also a graph of a function $V_2 \to V_1$. Therefore the matrix for $V$ in the charts $\Phi_i = \Phi_{S_i}$ for $i=1,2$ must be inverses of each other. Consequently the transition map between charts $\Phi_1$ and $\Phi_2$ is just the inversion map on $\GL_2(\R)$, which is smooth.\\

The dimension of $\Gr(4,2)$ is $4$. 
\section*{Problem 7}
Given four real numbers $a<b<c<d$, one constructs a smooth function $\B:\R\to \R$ such that $\B$ takes values between $0, 1$, is $1$ on $[b,c]$ and is $0$ outside $(a,d)$. We construct $\B$ as follows: let \begin{align*} f(x) = e^{1/x} & \text{ if } x>0\\ f(x) = 0 & \text{ if } x\leq 0. \end{align*} Then, $$g(x) = \frac{f(x)}{f(x) + f(1-x)}$$ is an increasing, smooth function which is $0$ for $x \leq 0$ and is $1$ for $x\geq 1$. Now define $$\B(x) = g\left(\frac{x-a}{b-a}\right)g\left(\frac{x-d}{c-d}\right).$$ We call such a $\B$ a bump function.

Given a box $D = \prod (a_i, d_i)$ in $\R^n$, and a smaller box $D' = \prod (b_i,c_i)$ inside $D$, one can construct a multivariable bump function $\B: D \to \R$ with similar properties. Indeed, consider $\B_i$ to be the bump function associated to $a_i < b_i < c_i < d_i$, and let $\B = \prod \B_i$. This smooth, multivariable bump function satisfies the following properties:

\begin{enumerate}
\item For $x \in D'$, $\B(x) = 1$.
\item For $x \not \in D$, $\B(x) = 0$.
\item For $x \in \R^n$, $0 \leq \B(x) \leq 1$.
\end{enumerate}

Given $p$ in $X$, choose a chart $\varphi: V \to \varphi(V) \subset \R^n$ around $p$. Then $\varphi(V)$ is an open subset of $\R^n$, so there are open boxes $D \supset D'$ with $D \subsetneq V$ around $\varphi(p)$ (choose $D$ such that the closure of $D$ is still contained in $V$). Let $U = \varphi^{-1}(D'), U' = \varphi^{-1}(D)$ and let $\B$ be a bump function satisfying $(1),(2)$ and $(3)$ above. Define $\beta: V \to \R$ by $\beta = \B \circ \varphi$. Then, extend $\beta$ to $X$ by declaring $\beta(x) = 0$ whenever $x\not \in V$. Since $\B$ is smooth in $\varphi(V)$, $\beta$ is smooth in $V$. Further, $\beta$ is $0$ outside the closure of $U'$. By construction of $D'$, the complement of the closure of $U'$ is an open set containing $V$. So, $\beta$ is smooth outside $V$ since it is a constant function on the open set $\overline U^c \supset V$. We have thus exhibited a smooth function $\beta$ on $X$ for which

\begin{enumerate}
\item $\beta(x) = 1$ for $x \in U$.
\item $\beta(x) = 0$ for $x \not \in V$.
\item $0\leq \beta(x) \leq 1$ for all $x$.
\end{enumerate}

For Problem 1(b), let $f$ be a function defined on open set $W$ which is smooth in atlas $\A_1$ but not smooth at $p$ in the atlas $\A_2$. Choose neighborhoods $U = U \cap W \subset V = V\cap W$ of $p$ and $\beta: X \to \R$ a bump function supported near $p$. Restrict $f$ to $V$ and consider the smooth (in atlas $\A_1$) $\beta f: X \to \R, x \mapsto \beta(x) f(x)$ for $x \in V$ and $x \mapsto 0$ for $x$ outside $V$. Then $\beta$ is not smooth in atlas $\A_2$ since $\beta f$ is the same as $f$ for a small enough neighborhood around $p$ (more precisely, $\beta f|_U = f|_U$). So we have a smooth function $g = \beta f$ defined on $X$ which is differentiable wrt one atlas but not the other. 

\end{document}