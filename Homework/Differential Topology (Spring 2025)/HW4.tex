\documentclass{amsart}

\usepackage{amsmath,amssymb,amsfonts,amscd}
\usepackage[all,cmtip]{xy}
\usepackage{enumerate}
\usepackage{ amssymb, latexsym, amsmath}
%\setcounter{MaxMatrixCols}{30}%
\usepackage{fullpage}
\usepackage{url}
\usepackage{hyperref}
\usepackage{ marvosym }
\usepackage{quiver}



\numberwithin{equation}{section}

%\usepackage{cjw-latex}

%\providecommand{\U}[1]{\protect\rule{.1in}{.1in}}

\theoremstyle{plain}
\newtheorem{theorem}{Theorem}[section]
\newtheorem*{thm}{Theorem}
\newtheorem{proposition}[theorem]{Proposition}
\newtheorem{lemma}[theorem]{Lemma}
\newtheorem{conjecture}[theorem]{Conjecture}
\newtheorem{corollary}[theorem]{Corollary}
\newtheorem{algorithm}[theorem]{Algorithm}
\newtheorem{axiom}[theorem]{Axiom}
\newtheorem{criterion}[theorem]{Criterion}
%\newtheorem{conjecture}[theorem]{Conjecture}
\newtheorem{wildconjecture}[theorem]{Wild Conjecture}


\theoremstyle{definition}
\newtheorem{definition}[theorem]{Definition}
\newtheorem*{dfn}{Definition}
\newtheorem{condition}[theorem]{Condition}
\newtheorem{example}[theorem]{Example}
\newtheorem{exercise}[theorem]{Exercise}
\newtheorem{notation}[theorem]{Notation}
\newtheorem{question}[theorem]{Question}
\newtheorem{problem}[theorem]{Problem}
\newtheorem{solution}[theorem]{Solution}




\theoremstyle{remark}
\newtheorem{remark}[theorem]{Remark}
\newtheorem{remarks}[theorem]{Remarks}
\newtheorem{summary}[theorem]{Summary}
\newtheorem{observation}[theorem]{Observation}
\newtheorem{conclusion}[theorem]{Conclusion}
\newtheorem{acknowledgement}[theorem]{Acknowledgement}
\newtheorem{case}[theorem]{Case}
\newtheorem{claim}[theorem]{Claim}



\makeatletter
\newcommand{\rmnum}[1]{\romannumeral #1}
\newcommand{\Rmnum}[1]{\expandafter\@slowromancap\romannumeral #1@}
\makeatother
\newcommand{\Aut}{\operatorname{Aut}}
\newcommand{\Ext}{\operatorname{Ext}}
\newcommand{\Tor}{\operatorname{Tor}}
\newcommand{\Z}{\mathbb{Z}}
\newcommand{\osum}{\oplus}
\newcommand{\aff}{\operatorname{Aff}}

\newcommand{\Id}{\operatorname{Id}}
\newcommand{\concat}{(\Id \to \Omega \Sigma)}
\newcommand{\eval}{(\Sigma\Omega \to \Id)}
\newcommand{\Q}{\mathbb{Q}}
\newcommand{\F}{\mathbb{F}}
\newcommand{\N}{\mathbb{N}}
\newcommand{\g}{\mathfrak{g}}
\newcommand{\gl}{\mathfrak{gl}}
\newcommand{\n}{\mathfrak{n}}
\newcommand{\h}{\mathfrak{h}}
\newcommand{\p}{\mathfrak{p}}
\newcommand{\q}{\mathfrak{q}}
\newcommand{\m}{\mathfrak{m}}
\newcommand{\e}{\mathfrak{e}}
\newcommand{\f}{\mathfrak{f}}
\newcommand{\V}{\mathscr{V}}
\newcommand{\D}{\mathscr{D}}
\renewcommand{\S}{\mathscr{S}}
\newcommand{\Sch}{\mathscr{C}}
\newcommand{\Y}{\mathscr{Y}}
\renewcommand{\u}{\mathfrak{u}}
\newcommand{\set}[2]{ \left\{ {#1} \, \left| \, {#2} \right\}\right.}
\newcommand{\A}{\mathbb{A}}
\newcommand{\X}{\mathcal{X}}
\renewcommand{\Im}{\operatorname{Im}}
\newcommand{\T}{\mathscr{T}}
\newcommand{\HH}{\mathscr{H}}
\newcommand{\PGL}{\operatorname{PGL}}
\newcommand{\righthookarrow}{\hookrightarrow}
\newcommand{\sign}{\operatorname{sign}}
\renewcommand{\_}[2]{\underbrace{#1}_{#2}}
\renewcommand{\^}[2]{\overbrace{#1}_{#2}}
\newcommand{\curverightarrow}{\curvearrowright}

\newcommand{\z}{\mathfrak{z}}
\renewcommand{\sl}{\mathfrak{sl}}
\newcommand{\R}{\mathbb{R}}
\newcommand{\RP}{\mathbb{RP}}
\renewcommand{\P}{\mathbb{P}}
\newcommand{\C}{\mathbb{C}}
\renewcommand{\H}{\mathcal{H}}
\newcommand{\Ind}{\operatorname{Ind}}
\newcommand{\Tr}{\operatorname{Tr}}
\newcommand{\Orb}{\mathcal{O}}
\renewcommand{\k}{\mbox{\Fontauri k}}
\newcommand{\Ad}{\operatorname{Ad}}
\newcommand{\ad}{\operatorname{ad}}
\newcommand{\Hom}{\operatorname{Hom}}
\newcommand{\Ker}{\operatorname{Ker}}
\newcommand{\End}{\operatorname{End}}
\newcommand{\supp}{\operatorname{supp}}
\newcommand{\Stab}{\mbox{Stab}}
\newcommand{\Lie}{\operatorname{Lie}}
\newcommand{\GL}{\operatorname{GL}}
\newcommand{\SL}{\operatorname{SL}}
\newcommand{\Sp}{\operatorname{Sp}}
\newcommand{\Mp}{\operatorname{Mp}}
\renewcommand{\a}{\mathfrak{a}}
\newcommand{\Wh}{\operatorname{Wh}}
\newcommand{\Span}{\operatorname{Span}}
\newcommand{\WF}{\operatorname{WF}}
\newcommand{\AC}{\operatorname{AC}}
\newcommand{\Lin}{\operatorname{Lin}}
\newcommand{\Diff}{\operatorname{Diff}}
\newcommand{\Gen}{\operatorname{Gen\, Hom}}
\newcommand{\Her}{\operatorname{Her}}
\newcommand{\pivot}{&}
\newcommand{\sgn}{\operatorname{sgn}}
\newcommand{\ASym}{\operatorname{ASym}}
\renewcommand{\d}{\mathrm{d}}
%\newcommand{C^{\infty}}{DS}
\newcommand{\vsp}{{\vspace{0.2in}}}
\renewcommand{\Ad}{\operatorname{Ad}}


\title{Notes}

\author{Sudharshan K V}
\begin{document}
\maketitle

\section*{Problem 1}
(a) If $f: X \to Y$ is a submersion, then it is locally surjective. Let $W$ be any open subset of $X$. For some $p$ in $W$, by local surjectivity, there are neighborhoods $U$ of $p$ and $V$ of $f(p)$ such that $f: U \to V$ is surjective. Choose a small enough neighborhood $V'$ around $f(p)$ such that $f^{-1}(V')$ is contained in $W$: one can do this by passing through coordinates. Then, $f(W)$ contains $f(f^{-1}(V')) = V'$, an open set around $f(p)$. So, $f(W)$ is open.  \\

(b) By (a), the image $f(X)$ is an open subset of $Y$. It is also compact since $f$ is continuous. Recall that a compact subspace of a Hausdorff space is closed. So $f(X)$ is an open and closed subset of $Y$. Since $X$ is non-empty $f(X)$ must be the whole codomain $Y$. 

\section*{Problem 2}
Since $X$ and $Y$ have the same dimension, (by preimage theorem) $f^{-1}(y)$ is a submanifold of $X$ of dimension $0$. So by definition, $f^{-1}(y)$ is discrete. It is also a closed subset of $X$ since $f$ is continuous, and hence is compact. A compact, discrete set must be finite.

Say $x_1, \dots, x_n$ be the points which map to $y$. Choose neighborhoods $U_i$ around $x_i$ and neighborhoods $V_i$ around $y$ such that $f: U_i \to V_i$ is a diffeomorphism for all $i$. Shrink the $U_i$ (and consequently the $V_i$) so that the $U_i$ are disjoint. One achieves this by using Hausdorff-ness, since $\{x_i\}$ is a finite set in $X$. Let $\tilde V = \cap_i V_i$, an open neighborhood of $y$, and shrink the $U_i$ so that $U_i$ are diffeomorphic to $\tilde V$. We therefore are now in the following setting:

\begin{itemize}
\item $\tilde V$ is a neighborhood of $y$.
\item $U_i$ are disjoint open subsets of $x_i$ which are homeomorphic to $\tilde V$ under $f$.
\end{itemize}

Notice that since $X$ is compact, so is the closed subset $X':= X \setminus \bigcup U_i$. So $f(X')$ is compact, and hence closed in $Y$. But $f(X')$ does not contain $y$ since $f^{-1}(y)$ is contained in $\bigcup U_i$. So there is a neighborhood $V$ around $y$ for which $f(X')$ does not meet $V$. Taking $V$ to be small enough to be contained in $\tilde V$, we have obtained a neighborhood of $y$ for which $f^{-1}(V)$ is contained in the union of the $U_i$. Shrink the $U_i$ further so that $f$ restricts to a diffeomorphism $f:U_i \to V$ for each $i$. Then the neighborhood $V$ along with the open sets $U_i$ satisfy:

\begin{itemize}
\item $f^{-1}(V)$ is the disjoint union $\bigsqcup U_i$ of a finite number of opens.
\item $f$ maps each $U_i$ diffeomorphicallly onto $V$. 
\end{itemize}

\section*{Problem 3}
(a) By defintion of a submanifold, there is a chart $\varphi_U:U \to \R^n$ around $p$ for which $$\varphi_U|_{U\cap S}: U\cap S \stackrel{\simeq}{\to} \R^k \cap U \subset \R^n.$$ The last inclusion is the first $k$-coordinate inclusion of Euclidean spaces. Let $\pi: U\subset \R^n \to \R^{n-k}$ be the projection onto the last $n-k$ coordinates. Then $U\cap \R^k$ is the level set $\pi^{-1}(0)$. So, $U\cap S$ is the level set of the function $f = \pi \circ \varphi_U: U \to \R^{n-k}$.

It is also clear that $Df$ is surjective everywhere on $U$ since $D\pi$ is surjective and $\varphi_U$ is a diffeomorphism. So in particular, $Df$ is surjective along $U\cap S$. \\

(b) Consider the inclusion $i:S\to \R^2$. Then one computes $Di$ in the standard chart as $Di = (D\pi_x|_S, D\pi_y|_S)$. But $i$ is an inclusion of a submanifold so it is an immersion. Hence $Di$ is non-zero, so one of $D\pi_x|_S$ or $D\pi_y|_S$ must be non-zero. But $\pi_{x,y}|_S$ are maps of $1$-dimensional manifolds, where having a non-zero derivative means the funcion is a local diffeomorphism by the inverse function theorem. \\

(c) Around the corner $(1,1)$ neither map $\pi_X$ or $\pi_Y$ is (locally) injective, so they cannot be local diffeomorphisms. Hence the square is not a submanifold of $\R^2$. 
\section*{Problem 4}
(a) We work in open subsets of $\R\P^3$. Consider the chart $U_0$ where the first coordinate $z_0$ is non-zero. Consider the smooth map $[1:a:b:c] \mapsto c - ab$. The derivative of this map in coordinates is $[*, *, 1]$, which is surjective. So the preimage of $0$, which is $U_0\cap H$ is a submanifold of $\R\P^3$. The argument for the other charts $U_{1,2,3}$ are similar, which shows that $U_i\cap H$ is a submanifold of $\R\P^3$. It follows that $H$ is a submanifold of $\R\P^3$. \\

(b) We will compute the derivatives of the map in the coordinate chart $U_0 \times U_0$ around $x_0\neq 0, y_0 \neq 0$. On the codomain take the chart $U_0$ again. In these charts, the map becomes $(a,b) \mapsto (b,a,ab)$. Clearly this map is smooth. We note that the derivative of this map is injective.

The image of this map lies in $H$. Furthermore, given some point $[z_0: z_1: z_2: z_3]$ in $U_0 \cap H$, the map $\sigma$ takes the point $[z_0:z_2],[z_0,z_1]$ to $[z_0^2: z_0z_1:z_0z_2:z_0z_3]$, where we have used $z_1z_2 = z_0z_3$. So the image of $\sigma$ contains $U_0\cap H$. The other charts have similar preimages, so image of $\sigma$ is $H$.

We next show that $\sigma$ is injective. Given the point $[z_0:\dots:z_3] = [x_0y_0: \dots :x_1y_1]$ in the image of $\sigma$, one can recover the points $[x_0:x_1]$ and $[y_0:y_1]$. WLOG suppose $z_0 \neq 0$. Then $[x_0:x_1] = [z_0: z_2]$ and $[y_0:y_1] = [z_0:z_1]$. Therefore the map is injective.

Recall that $\R\P^1$ is diffeomorphic to $S^1$. So we have an injective continuous map from a compact space $\R\P^1 \times \R\P^1$ to a Hausdorff space, and hence the map must be an embedding.\\

(c) The map $\sigma$ is a smooth embedding, and also an immersion. Therefore, $\R\P^1\times \R\P^1$ is diffeomorphic to its image $H$. Since $\R\P^1$ is diffeomorphic to $S^1$, we see that $H$ is diffeomorphic to the torus.

\section*{Problem 5}

We realize $\Sp_{2n}$ as a submanifold of $\GL_{2n}(\R)$. Any matrix $A$ which satisfies $A^T\Omega A = \Omega$ must be invertible which is clear by taking determinants. Also, any matrix of the form $A^T\Omega A$ is antisymmetric, owing to the fact that $\Omega^T = -\Omega$. Let $\ASym_{2n}(\R)$ denote the manifold of anti-symmetric matrices of size $2n$. An anti-symmetric matrix is determined by its strictly upper triangular entries, and hence $\ASym_{2n}(\R)$ has dimension $2n(2n-1)/2$. \\

Consider the map $F: \GL_{2n}(\R) \to \ASym_{2n}(\R)$ given by $A \mapsto A^T\Omega A$. This map is a polynomial map, and hence it is smooth. We compute the derivative in the usual way:
\begin{align*}
  DF_A(B) &= \frac{\d}{\d t} \left( F(A+tB) - F(A) \right) \\
          &= \frac{\d}{\d t} \left( A^T \Omega B + B^T \Omega A \right) t + O(t^2) \\
          &= A^T \Omega B + B^T \Omega A.
\end{align*}
We will show that for $F(A) = \Omega$, the map $DF_A$ is surjective onto the anti-symmetric matrices. (We have identified the tangent space of $\ASym_{2n}$ with the antisymmetric matrices.) Indeed, let $B = AC$ where $C$ is another matrix, written $
\begin{pmatrix}
  C_{11} & C_{12}\\
  C_{21} & C_{22}
\end{pmatrix}$ in blocks of $n\times n$ matrices. Then, $DF_A(B) = \Omega C + C^T \Omega$, which one computes by hand to be $$M_C: = 
\begin{pmatrix}
  C_{21} - C_{21}^T & C_{22} + C_{11}^T \\
  -C_{22}^T - C_{11} & -C_{12} + C_{12}^T
\end{pmatrix}.$$For any antisymmetric $2n\times 2n$ matrix $M'$, one can choose the blocks $C_{21}$ and $C_{12}$ so that the top right and bottom left $n\times n$ submatrices of $M_C$ and $M'$ coincide. Further, we may choose $C_{11}$ and $C_{22}$ freely so that $C_{11}^T + C_{22}$ is the top right $n\times n$ block of $M'$. The differential map is therefore surjective onto the antisymmetric matrices. By the preimage theorem, this means that $\Sp_{2n}(\R) = F^{-1}(\Omega)$ is a submanifold of $GL_{2n}(\R)$. The dimension of $\Sp_{2n}$ is $4n^2 - (2n^2 - n) = 2n^2 + n$.

The tangent space at identity is the kernel of $DF_\mathrm{id}$. Consulting the above calculations, these are matrices $C$ for which $C_{12}, C_{21}$ are symmetric, and $C_{22} + C_{11}^T = 0$. The tangent space at $A$ is the set of such matrices $C$ multiplied by $A$.

\section*{Problem 6}
(a) The multiplication map on $M_n$ is a polynomial map, and is hence smooth. The inversion map on $GL_n$ is given by $\det ^{-1} \mathrm{Adj}$, where $\mathrm{Adj}$ takes the adjoint of a matrix. $\mathrm{Adj}$ is a polynomial map, and $\det$ is a smooth map, so the inversion map is smooth (since we are away from $\det = 0$). Hence $\GL_n$ is a Lie group.

\begin{lemma}
  Let $i:Z \to Y$ be the inclusion of a submanifold $Z$ into $Y$. Let $f:X \to Z$ be a map. Then $f$ is smooth if and only if $i\circ f$ is smooth.
\end{lemma}
\begin{proof}
  One direction is obvious. Suppose $i\circ f$ is smooth. For $p \in X$, write $i$ in coordinates around $f(p)$ as the canonical inclusion of Euclidean space into a bigger Euclidean space. We can do this because $i$ is an immersion. Let $\pi$ be the projection which is the inverse of the inclusion of Euclidean spaces. Then, $f$ in coordinates is just $\pi \circ i\circ f$ in coordinates. The latter map is smooth since $\pi$ and $i\circ f$ are smooth. So $f$ is smooth. 
\end{proof}

Let $G$ be a submanifold and subgroup of $\GL_n$. Then consider multiplication and addition $\times: G\times G\to G$ and $[]^{-1}: G\to G$ as maps into $\GL_n$. Using the previous lemma and the fact that inversion and matrix multiplication are smooth, we see that $\times$ and $[]^{-1}$ are smooth maps, i.e. $G$ is a Lie group.\\

(b) For a curve $\gamma \in T_a\GL_n$, the derivative $DL_g|_a$ takes $\gamma$ to $g \gamma$. If we take standard charts of $M_n$ around the points $a, ga$, this map just takes a matrix $M$ to $gM$, since $g$ is a linear map, its derivative is itself).\\

(c) Let $i:G\to \GL_n$ be the inclusion map. Then we can identify $T_pG$ at any point $p$ with a subspace of $T_p\GL_n$. But $T_p\GL_n$ is identified with $M_n$ in the standard chart.

The multiplication by $g$ map on $G$ extends to the multiplication by $g$ map on $\GL_n$, and induces the following commutative diagram on the tangent spaces.
\[\begin{tikzcd}
	{T_e\GL_n} && {T_g\GL_n} \\
	\\
	{T_eG} && {T_gG}
	\arrow["{DL_g|_e}", from=1-1, to=1-3]
	\arrow["Di"{description}, from=3-1, to=1-1]
	\arrow["{DL_g|_e}", from=3-1, to=3-3]
	\arrow["Di"{description}, from=3-3, to=1-3]
      \end{tikzcd}.\]
The top map was computed to be multiplication by $g$ on $M_n$, so as subspaces of $M_n$, $T_gG$ is obtained by (left) multiplying all elements of $T_eG$ by $g$.\\

(d) Let $R_g$ denote right multiplication by $g^{-1}$. Then $C_g = L_gR_g$, and so we have $DC_g = DL_g DR_g$. By similar reasoning as (b), $DR_g$ is right multiplication by $g^{-1}$ on $M_n$, so we have $DC_g: \g \to \g$ to be the conjugation by $g$ map on $M_n$, where $\g$ is a subspace of $M_n$.\\

(e) We compute the derivative of the adjoint representation for $\GL_n$. We can identify the Lie algebra $\gl_n$ of $\GL_n$ with $M_n$ and we have $D\Ad_e: M_n \to \End(M_n)$. 
\begin{align*}
  D\Ad_e(X) &= \left(\frac{\d }{\d t}\right)_{t=0} \left (Y \mapsto (I+tX) Y(I+tX)^{-1}\right ) \\
            &= \left(Y \mapsto XY - Y\left(\frac{\d }{\d t}\right)_{t=0}(I+tX)^{-1}\right) \\
            &= Y \mapsto XY - YX.
\end{align*}
The last equality is due to the derivative of $t\mapsto (I+tX)^{-1}$ around $0$ being $-X$. This follows immediately from the product rule on $t\mapsto (I+tx)^{-1}(I+tX)$. We thus have $D\Ad_e(X)(Y) = XY - YX$.

For a Lie group $G \subset \GL_n$, we see that the adjoint representation of $G$ lifts to the adjoint representation of $\GL_n$. Therefore the map on the Lie algebra $\g$ induced by $\Ad_e$ must restrict from the map induced by $\Ad_e$ on $\gl_n$. Therefore, for $Y \in \g$, we have $D\Ad_e(X):\g \to \End \g$, $$D\Ad_e(X)(Y) = XY - YX.$$
\end{document}
