\documentclass{amsart}

\usepackage{amsmath,amssymb,amsfonts,amscd}
\usepackage[all,cmtip]{xy}
\usepackage{enumerate}
\usepackage{ amssymb, latexsym, amsmath}
%\setcounter{MaxMatrixCols}{30}%
\usepackage{fullpage}
\usepackage{url}
\usepackage{hyperref}
\usepackage{ marvosym }




\numberwithin{equation}{section}

%\usepackage{cjw-latex}

%\providecommand{\U}[1]{\protect\rule{.1in}{.1in}}

\theoremstyle{plain}
\newtheorem{theorem}{Theorem}[section]
\newtheorem*{thm}{Theorem}
\newtheorem{proposition}[theorem]{Proposition}
\newtheorem{lemma}[theorem]{Lemma}
\newtheorem{conjecture}[theorem]{Conjecture}
\newtheorem{corollary}[theorem]{Corollary}
\newtheorem{algorithm}[theorem]{Algorithm}
\newtheorem{axiom}[theorem]{Axiom}
\newtheorem{criterion}[theorem]{Criterion}
%\newtheorem{conjecture}[theorem]{Conjecture}
\newtheorem{wildconjecture}[theorem]{Wild Conjecture}


\theoremstyle{definition}
\newtheorem{definition}[theorem]{Definition}
\newtheorem*{dfn}{Definition}
\newtheorem{condition}[theorem]{Condition}
\newtheorem{example}[theorem]{Example}
\newtheorem{exercise}[theorem]{Exercise}
\newtheorem{notation}[theorem]{Notation}
\newtheorem{question}[theorem]{Question}
\newtheorem{problem}[theorem]{Problem}
\newtheorem{solution}[theorem]{Solution}




\theoremstyle{remark}
\newtheorem{remark}[theorem]{Remark}
\newtheorem{remarks}[theorem]{Remarks}
\newtheorem{summary}[theorem]{Summary}
\newtheorem{observation}[theorem]{Observation}
\newtheorem{conclusion}[theorem]{Conclusion}
\newtheorem{acknowledgement}[theorem]{Acknowledgement}
\newtheorem{case}[theorem]{Case}
\newtheorem{claim}[theorem]{Claim}



\makeatletter
\newcommand{\rmnum}[1]{\romannumeral #1}
\newcommand{\Rmnum}[1]{\expandafter\@slowromancap\romannumeral #1@}
\makeatother
\newcommand{\Aut}{\operatorname{Aut}}
\newcommand{\Ext}{\operatorname{Ext}}
\newcommand{\Tor}{\operatorname{Tor}}
\newcommand{\Z}{\mathbb{Z}}
\newcommand{\osum}{\oplus}
\newcommand{\aff}{\operatorname{Aff}}

\newcommand{\Id}{\operatorname{Id}}
\newcommand{\concat}{(\Id \to \Omega \Sigma)}
\newcommand{\eval}{(\Sigma\Omega \to \Id)}
\newcommand{\Q}{\mathbb{Q}}
\newcommand{\F}{\mathbb{F}}
\newcommand{\N}{\mathbb{N}}
\newcommand{\g}{\mathfrak{g}}
\newcommand{\n}{\mathfrak{n}}
\newcommand{\h}{\mathfrak{h}}
\newcommand{\p}{\mathfrak{p}}
\newcommand{\q}{\mathfrak{q}}
\newcommand{\m}{\mathfrak{m}}
\newcommand{\e}{\mathfrak{e}}
\newcommand{\f}{\mathfrak{f}}
\newcommand{\V}{\mathscr{V}}
\newcommand{\D}{\mathscr{D}}
\renewcommand{\S}{\mathscr{S}}
\newcommand{\Sch}{\mathscr{C}}
\newcommand{\Y}{\mathscr{Y}}
\renewcommand{\u}{\mathfrak{u}}
\newcommand{\set}[2]{ \left\{ {#1} \, \left| \, {#2} \right\}\right.}
\newcommand{\A}{\mathbb{A}}
\newcommand{\X}{\mathcal{X}}
\renewcommand{\Im}{\operatorname{Im}}
\newcommand{\T}{\mathscr{T}}
\newcommand{\HH}{\mathscr{H}}
\newcommand{\PGL}{\operatorname{PGL}}
\newcommand{\righthookarrow}{\hookrightarrow}
\newcommand{\sign}{\operatorname{sign}}
\renewcommand{\_}[2]{\underbrace{#1}_{#2}}
\renewcommand{\^}[2]{\overbrace{#1}_{#2}}
\newcommand{\curverightarrow}{\curvearrowright}

\newcommand{\z}{\mathfrak{z}}
\renewcommand{\sl}{\mathfrak{sl}}
\newcommand{\R}{\mathbb{R}}
\newcommand{\RP}{\mathbb{RP}}
\renewcommand{\P}{\mathbb{P}}
\newcommand{\C}{\mathbb{C}}
\renewcommand{\H}{\mathcal{H}}
\newcommand{\Ind}{\operatorname{Ind}}
\newcommand{\Tr}{\operatorname{Tr}}
\newcommand{\Orb}{\mathcal{O}}
\renewcommand{\k}{\mbox{\Fontauri k}}
\newcommand{\Ad}{\operatorname{Ad}}
\newcommand{\ad}{\operatorname{ad}}
\newcommand{\Hom}{\operatorname{Hom}}
\newcommand{\Ker}{\operatorname{Ker}}
\newcommand{\End}{\operatorname{End}}
\newcommand{\supp}{\operatorname{supp}}
\newcommand{\Stab}{\mbox{Stab}}
\newcommand{\Lie}{\operatorname{Lie}}
\newcommand{\GL}{\operatorname{GL}}
\newcommand{\SL}{\operatorname{SL}}
\newcommand{\Sp}{\operatorname{Sp}}
\newcommand{\Mp}{\operatorname{Mp}}
\renewcommand{\a}{\mathfrak{a}}
\newcommand{\Wh}{\operatorname{Wh}}
\newcommand{\Span}{\operatorname{Span}}
\newcommand{\WF}{\operatorname{WF}}
\newcommand{\AC}{\operatorname{AC}}
\newcommand{\Lin}{\operatorname{Lin}}
\newcommand{\Diff}{\operatorname{Diff}}
\newcommand{\Gen}{\operatorname{Gen\, Hom}}
\newcommand{\Her}{\operatorname{Her}}
\newcommand{\pivot}{&}
\newcommand{\sgn}{\operatorname{sgn}}
%\newcommand{C^{\infty}}{DS}
\newcommand{\vsp}{{\vspace{0.2in}}}



\title{Notes}

\author{Sudharshan K V}
\begin{document}
\maketitle

\section*{Problem 1}
(a) If $f: X \to Y$ is a submersion, then it is locally surjective. Let $W$ be any open subset of $X$. For some $p$ in $W$, by local surjectivity, there are neighborhoods $U$ of $p$ and $V$ of $f(p)$ such that $f: U \to V$ is surjective. Choose a small enough neighborhood $V'$ around $f(p)$ such that $f^{-1}(V')$ is contained in $W$: one can do this by passing through coordinates. Then, $f(W)$ contains $f(f^{-1}(V')) = V'$, an open set around $f(p)$. So, $f(W)$ is open.  \\

(b) By (a), the image $f(X)$ is an open subset of $Y$. It is also compact since $f$ is continuous. Recall that a compact subspace of a Hausdorff space is closed. So $f(X)$ is an open and closed subset of $Y$. Since $X$ is non-empty $f(X)$ must be the whole codomain $Y$. 

\section*{Problem 2}
Since $X$ and $Y$ have the same dimension, (by preimage theorem) $f^{-1}(y)$ is a submanifold of $X$ of dimension $0$. So by definition, $f^{-1}(y)$ is discrete. It is also a closed subset of $X$ since $f$ is continuous, and hence is compact. A compact, discrete set must be finite.

Say $x_1, \dots, x_n$ be the points which map to $y$. Choose neighborhoods $U_i$ around $x_i$ and neighborhoods $V_i$ around $y$ such that $f: U_i \to V_i$ is a diffeomorphism for all $i$. Shrink the $U_i$ (and consequently the $V_i$) so that the $U_i$ are disjoint. One achieves this by using Hausdorff-ness, since $\{x_i\}$ is a finite set in $X$. Let $\tilde V = \cap_i V_i$, an open neighborhood of $y$, and shrink the $U_i$ so that $U_i$ are diffeomorphic to $\tilde V$. Further assume that $\tilde V$ is completely within a chart around $y$ of $Y$. We therefore are now in the following setting:

\begin{itemize}
\item $\tilde V$ is a neighborhood of $y$ which is homeomorphic to an open subset of Euclidean space.
\item $U_i$ are disjoint open subsets of $x_i$ which are homeomorphic to $\tilde V$ under $f$.
\end{itemize}

We claim that we can shrink $\tilde V$ so that $f^{-1}(\tilde V)$ is contained in the union of the $U_i$. Suppose otherwise, that there are points $p_i$ in $X':= X\setminus \cup U_i$ (a closed hence compact subset of $X$) for which $f(p_i)$ converges to $y$. Then compactness (hence sequential compactness)\footnote{For a Hausdorff, second countable topological space, compactness is equivalent to sequential compactness.} for $X'$ gives a subsequence of the $p_i$ which converges to some $p$ in $X'$. Clearly, $f(p)$ must be $y$ by continuity of $f$. But this violates the fact that $f^{-1}(y)$ is contained in $\cup U_i$. Therefore, there must be a small enough open subset $V$ of $\tilde V$ for which $f^{-1}(V)$ is contained in the $\cup U_i$. Replacing $U_i$ by the smaller open sets $f^{-1}(V) \cap U_i$ gives us the required finite-sheeted, covering map around $y$. 

\section*{Problem 3}

\section*{Problem 4}
(a) We work in open subsets of $\R\P^3$. Consider the chart $U_0$ where the first coordinate $z_0$ is non-zero. Consider the smooth map $[1:a:b:c] \mapsto c - ab$. The derivative of this map in coordinates is $[*, *, 1]$, which is surjective. So the preimage of $0$, which is $U_0\cap H$ is a submanifold of $\R\P^3$. The argument for the other charts $U_{1,2,3}$ are similar, which shows that $U_i\cap H$ is a submanifold of $\R\P^3$. It follows that $H$ is a submanifold of $\R\P^3$. \\

(b) We will compute the derivatives of the map in the coordinate chart $U_0 \times U_0$ around $x_0\neq 0, y_0 \neq 0$. On the codomain take the chart $U_0$ again. In these charts, the map becomes $(a,b) \mapsto (b,a,ab)$. Clearly this map is smooth. The image of this map lies in $H$. Furthermore, given some point $[z_0: z_1: z_2: z_3]$ in $U_0 \cap H$, the map $\sigma$ takes the point $[z_0:z_2],[z_0,z_1]$ to $[z_0^2: z_0z_1:z_0z_2:z_0z_3]$, where we have used $z_1z_2 = z_0z_3$. So the image of $\sigma$ contains $U_0\cap H$. The other charts have similar preimages, so image of $\sigma$ is $H$. 

\end{document}
