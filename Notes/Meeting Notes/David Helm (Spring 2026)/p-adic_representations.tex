\documentclass{amsart}

\usepackage{amsmath,amssymb,amsfonts,amscd}
\usepackage[all,cmtip]{xy}
\usepackage{enumerate}
\usepackage{ amssymb, latexsym, amsmath}
%\setcounter{MaxMatrixCols}{30}%
\usepackage{fullpage}
\usepackage{url}
\usepackage{hyperref}
\usepackage{ marvosym }
\usepackage{quiver}



\numberwithin{equation}{section}

%\usepackage{cjw-latex}

%\providecommand{\U}[1]{\protect\rule{.1in}{.1in}}

\theoremstyle{plain}
\newtheorem{theorem}{Theorem}[section]
\newtheorem*{thm}{Theorem}
\newtheorem{proposition}[theorem]{Proposition}
\newtheorem{lemma}[theorem]{Lemma}
\newtheorem{conjecture}[theorem]{Conjecture}
\newtheorem{corollary}[theorem]{Corollary}
\newtheorem{algorithm}[theorem]{Algorithm}
\newtheorem{axiom}[theorem]{Axiom}
\newtheorem{criterion}[theorem]{Criterion}
%\newtheorem{conjecture}[theorem]{Conjecture}
\newtheorem{wildconjecture}[theorem]{Wild Conjecture}


\theoremstyle{definition}
\newtheorem{definition}[theorem]{Definition}
\newtheorem*{dfn}{Definition}
\newtheorem{condition}[theorem]{Condition}
\newtheorem{example}[theorem]{Example}
\newtheorem{exercise}[theorem]{Exercise}
\newtheorem{notation}[theorem]{Notation}
\newtheorem{question}[theorem]{Question}
\newtheorem{problem}[theorem]{Problem}
\newtheorem{solution}[theorem]{Solution}




\theoremstyle{remark}
\newtheorem{remark}[theorem]{Remark}
\newtheorem{remarks}[theorem]{Remarks}
\newtheorem{summary}[theorem]{Summary}
\newtheorem{observation}[theorem]{Observation}
\newtheorem{conclusion}[theorem]{Conclusion}
\newtheorem{acknowledgement}[theorem]{Acknowledgement}
\newtheorem{case}[theorem]{Case}
\newtheorem{claim}[theorem]{Claim}



\makeatletter
\newcommand{\rmnum}[1]{\romannumeral #1}
\newcommand{\Rmnum}[1]{\expandafter\@slowromancap\romannumeral #1@}
\makeatother
\newcommand{\Aut}{\operatorname{Aut}}
\newcommand{\Ext}{\operatorname{Ext}}
\newcommand{\Tor}{\operatorname{Tor}}
\newcommand{\Z}{\mathbb{Z}}
\newcommand{\osum}{\oplus}
\newcommand{\aff}{\operatorname{Aff}}

\newcommand{\Id}{\operatorname{Id}}
\newcommand{\concat}{(\Id \to \Omega \Sigma)}
\newcommand{\eval}{(\Sigma\Omega \to \Id)}
\newcommand{\Q}{\mathbb{Q}}
\newcommand{\F}{\mathbb{F}}
\newcommand{\N}{\mathbb{N}}
\newcommand{\g}{\mathfrak{g}}
\newcommand{\n}{\mathfrak{n}}
\newcommand{\h}{\mathfrak{h}}
\newcommand{\p}{\mathfrak{p}}
\newcommand{\q}{\mathfrak{q}}
\newcommand{\m}{\mathfrak{m}}
\newcommand{\e}{\mathfrak{e}}
\newcommand{\f}{\mathfrak{f}}
\newcommand{\V}{\mathscr{V}}
\newcommand{\D}{\mathscr{D}}
\renewcommand{\S}{\mathscr{S}}
\newcommand{\Sch}{\mathscr{C}}
\newcommand{\Y}{\mathscr{Y}}
\renewcommand{\u}{\mathfrak{u}}
\newcommand{\set}[2]{ \left\{ {#1} \, \left| \, {#2} \right\}\right.}
\newcommand{\A}{\mathbb{A}}
\newcommand{\X}{\mathcal{X}}
\renewcommand{\Im}{\operatorname{Im}}
\newcommand{\T}{\mathscr{T}}
\newcommand{\HH}{\mathscr{H}}
\newcommand{\PGL}{\operatorname{PGL}}
\newcommand{\righthookarrow}{\hookrightarrow}
\newcommand{\sign}{\operatorname{sign}}
\renewcommand{\_}[2]{\underbrace{#1}_{#2}}
\renewcommand{\^}[2]{\overbrace{#1}_{#2}}
\newcommand{\curverightarrow}{\curvearrowright}

\newcommand{\z}{\mathfrak{z}}
\renewcommand{\sl}{\mathfrak{sl}}
\newcommand{\R}{\mathbb{R}}
\newcommand{\RP}{\mathbb{RP}}
\renewcommand{\P}{\mathbb{P}}
\newcommand{\C}{\mathbb{C}}
\renewcommand{\H}{\mathcal{H}}
\newcommand{\Ind}{\operatorname{Ind}}
\newcommand{\Tr}{\operatorname{Tr}}
\newcommand{\Orb}{\mathcal{O}}
\renewcommand{\k}{\mbox{\Fontauri k}}
\newcommand{\Ad}{\operatorname{Ad}}
\newcommand{\ad}{\operatorname{ad}}
\newcommand{\Hom}{\operatorname{Hom}}
\newcommand{\Ker}{\operatorname{Ker}}
\newcommand{\End}{\operatorname{End}}
\newcommand{\supp}{\operatorname{supp}}
\newcommand{\Stab}{\mbox{Stab}}
\newcommand{\Lie}{\operatorname{Lie}}
\newcommand{\GL}{\operatorname{GL}}
\newcommand{\SL}{\operatorname{SL}}
\newcommand{\Sp}{\operatorname{Sp}}
\newcommand{\Mp}{\operatorname{Mp}}
\renewcommand{\a}{\mathfrak{a}}
\newcommand{\Wh}{\operatorname{Wh}}
\newcommand{\Span}{\operatorname{Span}}
\newcommand{\WF}{\operatorname{WF}}
\newcommand{\AC}{\operatorname{AC}}
\newcommand{\Lin}{\operatorname{Lin}}
\newcommand{\Diff}{\operatorname{Diff}}
\newcommand{\Gen}{\operatorname{Gen\, Hom}}
\newcommand{\Her}{\operatorname{Her}}
\newcommand{\pivot}{&}
\newcommand{\sgn}{\operatorname{sgn}}
%\newcommand{C^{\infty}}{DS}
\newcommand{\vsp}{{\vspace{0.2in}}}



\title{$p$-adic Reps, Spring 2026 Meeting Notes}

\author{Sudharshan K V}
\begin{document}
\maketitle

\section{Meeting 1}
Smoothness is the natural condition to impose on groups which are locally compact and totally disconnected - topologies on $\C$ and profinite groups are very different. Continuity is a weaker condition, for characters continuity and smoothness are the same. For discrete modules, the notions of continuity and smoothness agree.\\

It is hard to construct finite dimensional non-smooth representations. One may look at $C(G)$, the space of functions on $G$. This is very non-smooth. Perhaps looking at this can produce exact sequences of abstract representations where taking the smooth vectors wouldn't be right exact.\\

Restriction of an induced representation is nice for normal subgroups: $\Ind_H^G(V)|_H = \osum_{g\in G/H} V^g$. In general, this is the subject of Mackey theory. \\

Invariants and coinvaraints are isomorphic for a representation of a compact group.

\section{Meeting 2}
[Before meeting] Smooth duals are just usual duals when looking at compact-invariant subspaces. Admissibility implies reflexivity. Exactness of representations is the same as exactness at each compact-invariant level. Seems to be powerful.\\

Measures. Haar measure can be used to integrate some functions which are not compactly supported - there are some things which I need to verify - why left Haar measure and left invariant functions?. Integration of vector valued functions - can be done because functionals separate different vectors.\\

Modular character of a group may not be the same as the modular character of the subgroup. However if the subgroup is open they agree. There is a characterization in Prop, 3.4. For open subgroups, this is evident from Frobenius reciprocity. Frobenius reciprocity (for compact induction) is not valid for non-open subgroups in general.\\

\subsection{Meeting Notes}
Some computations on integration of left compact-invariant functions not necessarily compactly supported. It is important that the function and the Haar measure are both left invariant for the resulting integration to be left invariant.\\

Integration of locally constant functions are just sums (compact support - finite sums). Integrals are easier to manipulate in general; we see this later when computing the product of Hecke operators, convolution is an integration.\\

Remark on Schur's lemma: General form - if dimension of irrep $<$ cardinality of (algebraically closed) base field then Schur's lemma holds (same proof). But in the case of representations of $p$-adic groups over $\bar \F_\ell$ ($\ell \neq p$), Schur's lemma still holds (technical reasons?). Not sure about mod-$p$ representations of $p$-adic groups.

\section{Meeting 3}

[Before meeting] I did not do much. In the proof of Prop 3.4, they prove surjectivity of some kind of averaging map from $C_c^\infty(G) \to C_c^\infty(H\backslash G, \theta)$ by considering $K$-invariant subspaces of both sides. Then, LHS spanned by indicator functions of $gK$, while RHS is spanned by certain functions supported on the double coset $HgK$. I am not sure why the image of $\mathbf 1_{gK}$ is non-trivial when there is a non-zero function supported on $HgK$ in the RHS.\\

Hecke algebras seem powerful. $\H(G//K)$ is unital associative algebra; $V^K$ is a module over $\H(G//K)$. $\H(G) = \cup_K \H(G//K)$. The limit of the units in the small Hecke algebras as $K$ becomes smaller tends to the delta function at the identity, which is not an element of $\H(G)$ unless $G$ is discrete - in which case it serves as a unit $(\H(G) = \H(G//\{e\})$.\\

Is there a relation between this Hecke algebra and the archimedean Hecke algebra of distributions supported at the identity? Relation deeper than ``[adjectives] Representations of group = [adjectives] Representations of modules''?\\

Maybe further, relation between Hecke operators on modular forms and the local ones? How do you understand Hecke operator on modular forms in this way? Can you see local Hecke operator as some kind of averaging?\\

\subsection{Meeting Notes}

The proof of 3.4 needs to be elaborated. Say $\phi$ is a non-zero function in $C_c^\infty(H\backslash G, \theta)^K$ supported on double coset $HgK$, with $\phi(g) \neq 0$. Consider the indicator function $\mathbf 1_{gK}$ in $C_c^\infty(G)^K$. Compute the image of this function to get \[\tilde{\mathbf 1}_{gK}(g) = \int_{H\cap gKg^{-1}} \theta \delta_H(h)^{-1} dh.\]
  But, note that for $h \in H \cap gKg^{-1}$, one has \[\theta(h)\phi(g) = \phi(hg) = \phi(gk) = \phi(g),\] so $\theta$ is trivial on this group. So the integral in the image of $\mathbf 1_{gK}$ is non-zero (what's left is the modular character, and the integral evaluates to the volume of $gKg^{-1}\cap H$ in the right Haar measure on $H$).\\

  
\textbf{Spherical Hecke Algebras.} Let $f$ be a modular form, level $N$. Let $\pi_f$ be the corresponding automorphic form on $\GL_2$, and let $\pi$ be the corresponding automorphic representation. There is a Hecke operator $T_p \in \End (S_k(\Gamma_0(N))$, for $p\nmid N$. Let $G = \GL_2(\Q_p), K = \GL_2(\Z_p)$. Then the spherical Hecke algebra $\H(G//K)$ is spanned by $\mathbf 1_{KgK}$, for $g \in K\backslash G/K = \cup_{i>j} K\begin{pmatrix}
    p^i & 0 \\ 0 & p^j
\end{pmatrix} K.$ (Cartan decomposition).
The operator corresponding to $\mathbf 1_{K\left(\begin{smallmatrix}p&0\\0&1\end{smallmatrix}\right)K}$ is denoted $T_p$, and one has $T_p *\pi_f = \pi_{T_p(f)}$. Note that $\pi_f \in \pi^K$, which is a $\H(G//K)$-module. There is an algebra isomorphism of $\H$ and the subalgebra generated by $T_p$ and $\langle p \rangle$ in $\End(S_k(\Gamma_0(N))$.

For archimedean representations, one does not have $K$-invariance for compact $K$, but one does have $K$-finiteness.\\

\textbf{One should think of the space of double cosets as some kind of distances on $G$. Building a tree from $G,K$. Functions on $G/K$ are functions on lattices. Hecke action on functions is the same as averaging over lattices. I will expand more on this.}

\section{Meeting 4,5,6}

Dimension of Hom space between two principal series is $0,1$ or $2$. This seems to use both sides of the adjunction between Ind and Res for finite groups. Can you show $\Hom_B(\Ind_T^B(\chi^w), \xi)$ is one dimensional otherwise in more general situations? \\
Then complete reducibility and Schur's lemma to conclude principal series are irreducible.\\

Irreducible representation is called cuspidal if it does not contain the trivial character of $N$. The term \emph{cuspidal} may be because the sum of character values over $N$ is $0$ for a non-trivial character. In general, the integral over $[N]$ is supposed to represent the constant term. \\

Where did the construction of cuspidal representations come from? Why does the non-split torus $E$ enter the picture? What about $ZN$?\\

Whittaker models: $\Ind_N^G \psi$ for $\psi$ non-trivial contains all the irreducible representations of $G$, except twists of the determinant. One can see this using the classification of representations and Frobenius reciprocity, computing characters on $N$. An embedding $\sigma \hookrightarrow \Ind_N^G \psi$ for an irrep $\sigma$ is called a Whittaker model. The Hom spaces are one dimensional, so one has ``uniqueness of Whittaker model''. Almost every irrep of $G$ has a Whittaker model.

\subsection{Email exchange}

Construction of cuspidal representations of reductive groups over finite field - Deligne-Lusztig theory. For $GL_2$ the space of Whittaker functionals $\Ind_{ZN}^G \theta$ contains all the generic (hence cuspidal) irreducible representations with specific central character $(\theta)$. The complement of this space is the representation induced from the non-split torus - this is specific to $GL_2$. \\

To show existence of Whittaker model. Let $\sigma$ be an irrep of $G$ containing a non-trivial character of $N$. Then conjugating by Borel element shows that $\sigma$ contains $\psi$. Non-trival $N$-map $\sigma \to \psi$ gives an embedding $\sigma \to \Ind_N^G \psi$, i.e. a Whittaker model of $\sigma$. Multiplicity one is generality.\\
If $\sigma$ doesn't contain $\psi$, then conjugates of $N$ act trivially on $\sigma$, and so does every conjugate of $N$. These conjugates generate $\SL_2(\mathbb F_q)$. So $\sigma$ factors through the determinant.

\subsection{Jacquet Functors}

$V$ to $V_N$ from $G$ reps to $T$ reps. Jacquet functor is the adjoint of induction from Borel after inflation. \\

For $\GL_2$ over finite fields. For irrep $V$ of $G$, we have a characterization $V_N$ is non-zero iff $V$ is a subrepresentation of principal series. A representation is in principal series iff it contains the trivial character of $N$, i.e. there is $N$-fixed vector. So $V^N \neq 0$ iff $V_N \neq 0$.\\

For local fields, $V^N \neq 0$ means that the irrep is one dimensional, factoring through the determinant. So we have to use $V_N \neq 0$ to characterize subrepresentations of principal series. Finite dimensional representations have $N$-fixed vectors because a family of commuting operators have a common eigenvector, and $N$ acts unipotently so eigenvalues are $1$. Therefore, finite dimensional irreps of $G$ are one dimensional. \\

I think the Jacquet functor of a modular form is supposed to represent the constant term: if $f$ vanishes in $\pi_N$ for all $N$, where $\pi$ is the automorphic representation generated by $f$, then $f$ is a cusp form.
\end{document}
