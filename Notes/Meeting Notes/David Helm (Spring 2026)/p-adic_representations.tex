\documentclass{amsart}

\usepackage{amsmath,amssymb,amsfonts,amscd}
\usepackage[all,cmtip]{xy}
\usepackage{enumerate}
\usepackage{ amssymb, latexsym, amsmath}
%\setcounter{MaxMatrixCols}{30}%
\usepackage{fullpage}
\usepackage{url}
\usepackage{hyperref}
\usepackage{ marvosym }
\usepackage{quiver}



\numberwithin{equation}{section}

%\usepackage{cjw-latex}

%\providecommand{\U}[1]{\protect\rule{.1in}{.1in}}

\theoremstyle{plain}
\newtheorem{theorem}{Theorem}[section]
\newtheorem*{thm}{Theorem}
\newtheorem{proposition}[theorem]{Proposition}
\newtheorem{lemma}[theorem]{Lemma}
\newtheorem{conjecture}[theorem]{Conjecture}
\newtheorem{corollary}[theorem]{Corollary}
\newtheorem{algorithm}[theorem]{Algorithm}
\newtheorem{axiom}[theorem]{Axiom}
\newtheorem{criterion}[theorem]{Criterion}
%\newtheorem{conjecture}[theorem]{Conjecture}
\newtheorem{wildconjecture}[theorem]{Wild Conjecture}


\theoremstyle{definition}
\newtheorem{definition}[theorem]{Definition}
\newtheorem*{dfn}{Definition}
\newtheorem{condition}[theorem]{Condition}
\newtheorem{example}[theorem]{Example}
\newtheorem{exercise}[theorem]{Exercise}
\newtheorem{notation}[theorem]{Notation}
\newtheorem{question}[theorem]{Question}
\newtheorem{problem}[theorem]{Problem}
\newtheorem{solution}[theorem]{Solution}




\theoremstyle{remark}
\newtheorem{remark}[theorem]{Remark}
\newtheorem{remarks}[theorem]{Remarks}
\newtheorem{summary}[theorem]{Summary}
\newtheorem{observation}[theorem]{Observation}
\newtheorem{conclusion}[theorem]{Conclusion}
\newtheorem{acknowledgement}[theorem]{Acknowledgement}
\newtheorem{case}[theorem]{Case}
\newtheorem{claim}[theorem]{Claim}



\makeatletter
\newcommand{\rmnum}[1]{\romannumeral #1}
\newcommand{\Rmnum}[1]{\expandafter\@slowromancap\romannumeral #1@}
\makeatother
\newcommand{\Aut}{\operatorname{Aut}}
\newcommand{\Ext}{\operatorname{Ext}}
\newcommand{\Tor}{\operatorname{Tor}}
\newcommand{\Z}{\mathbb{Z}}
\newcommand{\osum}{\oplus}
\newcommand{\aff}{\operatorname{Aff}}

\newcommand{\Id}{\operatorname{Id}}
\newcommand{\concat}{(\Id \to \Omega \Sigma)}
\newcommand{\eval}{(\Sigma\Omega \to \Id)}
\newcommand{\Q}{\mathbb{Q}}
\newcommand{\F}{\mathbb{F}}
\newcommand{\N}{\mathbb{N}}
\newcommand{\g}{\mathfrak{g}}
\newcommand{\n}{\mathfrak{n}}
\newcommand{\h}{\mathfrak{h}}
\newcommand{\p}{\mathfrak{p}}
\newcommand{\q}{\mathfrak{q}}
\newcommand{\m}{\mathfrak{m}}
\newcommand{\e}{\mathfrak{e}}
\newcommand{\f}{\mathfrak{f}}
\newcommand{\V}{\mathscr{V}}
\newcommand{\D}{\mathscr{D}}
\renewcommand{\S}{\mathscr{S}}
\newcommand{\Sch}{\mathscr{C}}
\newcommand{\Y}{\mathscr{Y}}
\renewcommand{\u}{\mathfrak{u}}
\newcommand{\set}[2]{ \left\{ {#1} \, \left| \, {#2} \right\}\right.}
\newcommand{\A}{\mathbb{A}}
\newcommand{\X}{\mathcal{X}}
\renewcommand{\Im}{\operatorname{Im}}
\newcommand{\T}{\mathscr{T}}
\newcommand{\HH}{\mathscr{H}}
\newcommand{\PGL}{\operatorname{PGL}}
\newcommand{\righthookarrow}{\hookrightarrow}
\newcommand{\sign}{\operatorname{sign}}
\renewcommand{\_}[2]{\underbrace{#1}_{#2}}
\renewcommand{\^}[2]{\overbrace{#1}_{#2}}
\newcommand{\curverightarrow}{\curvearrowright}

\newcommand{\z}{\mathfrak{z}}
\renewcommand{\sl}{\mathfrak{sl}}
\newcommand{\R}{\mathbb{R}}
\newcommand{\RP}{\mathbb{RP}}
\renewcommand{\P}{\mathbb{P}}
\newcommand{\C}{\mathbb{C}}
\renewcommand{\H}{\mathcal{H}}
\newcommand{\Ind}{\operatorname{Ind}}
\newcommand{\Tr}{\operatorname{Tr}}
\newcommand{\Orb}{\mathcal{O}}
\renewcommand{\k}{\mbox{\Fontauri k}}
\newcommand{\Ad}{\operatorname{Ad}}
\newcommand{\ad}{\operatorname{ad}}
\newcommand{\Hom}{\operatorname{Hom}}
\newcommand{\Ker}{\operatorname{Ker}}
\newcommand{\End}{\operatorname{End}}
\newcommand{\supp}{\operatorname{supp}}
\newcommand{\Stab}{\mbox{Stab}}
\newcommand{\Lie}{\operatorname{Lie}}
\newcommand{\GL}{\operatorname{GL}}
\newcommand{\SL}{\operatorname{SL}}
\newcommand{\Sp}{\operatorname{Sp}}
\newcommand{\Mp}{\operatorname{Mp}}
\renewcommand{\a}{\mathfrak{a}}
\newcommand{\Wh}{\operatorname{Wh}}
\newcommand{\Span}{\operatorname{Span}}
\newcommand{\WF}{\operatorname{WF}}
\newcommand{\AC}{\operatorname{AC}}
\newcommand{\Lin}{\operatorname{Lin}}
\newcommand{\Diff}{\operatorname{Diff}}
\newcommand{\Gen}{\operatorname{Gen\, Hom}}
\newcommand{\Her}{\operatorname{Her}}
\newcommand{\pivot}{&}
\newcommand{\sgn}{\operatorname{sgn}}
%\newcommand{C^{\infty}}{DS}
\newcommand{\vsp}{{\vspace{0.2in}}}



\title{$p$-adic Reps, Spring 2026 Meeting Notes}

\author{Sudharshan K V}
\begin{document}
\maketitle

\section{Meeting 1}
Smoothness is the natural condition to impose on groups which are locally compact and totally disconnected - topologies on $\C$ and profinite groups are very different. Continuity is a weaker condition, for characters continuity and smoothness are the same. For discrete modules, the notions of continuity and smoothness agree.\\

It is hard to construct finite dimensional non-smooth representations. One may look at $C(G)$, the space of functions on $G$. This is very non-smooth. Perhaps looking at this can produce exact sequences of abstract representations where taking the smooth vectors wouldn't be right exact.\\

Restriction of an induced representation is nice for normal subgroups: $\Ind_H^G(V)|_H = \osum_{g\in G/H} V^g$. In general, this is the subject of Mackey theory. \\

Invariants and coinvaraints are isomorphic for a representation of a compact group.

\section{Meeting 2}
[Before meeting] Smooth duals are just usual duals when looking at compact-invariant subspaces. Admissibility implies reflexivity. Exactness of representations is the same as exactness at each compact-invariant level. Seems to be powerful.\\

Measures. Haar measure can be used to integrate some functions which are not compactly supported - there are some things which I need to verify - why left Haar measure and left invariant functions?. Integration of vector valued functions - can be done because functionals separate different vectors.\\

Modular character of a group may not be the same as the modular character of the subgroup. However if the subgroup is open they agree. There is a characterization in Prop, 3.4. For open subgroups, this is evident from Frobenius reciprocity. Frobenius reciprocity (for compact induction) is not valid for non-open subgroups in general.\\

\subsection{Meeting Notes}
Some computations on integration of left compact-invariant functions not necessarily compactly supported. It is important that the function and the Haar measure are both left invariant for the resulting integration to be left invariant.\\

Integration of locally constant functions are just sums (compact support - finite sums). Integrals are easier to manipulate in general; we see this later when computing the product of Hecke operators, convolution is an integration.\\

Remark on Schur's lemma: General form - if dimension of irrep $<$ cardinality of (algebraically closed) base field then Schur's lemma holds (same proof). But in the case of representations of $p$-adic groups over $\bar \F_\ell$ ($\ell \neq p$), Schur's lemma still holds (technical reasons?). Not sure about mod-$p$ representations of $p$-adic groups.

\section{Meeting 3}

[Before meeting] I did not do much. In the proof of Prop 3.4, they prove surjectivity of some kind of averaging map from $C_c^\infty(G) \to C_c^\infty(H\backslash G, \theta)$ by considering $K$-invariant subspaces of both sides. Then, LHS spanned by indicator functions of $gK$, while RHS is spanned by certain functions supported on the double coset $HgK$. I am not sure why the image of $\mathbf 1_{gK}$ is non-trivial when there is a non-zero function supported on $HgK$ in the RHS.\\

Hecke algebras seem powerful. $\H(G//K)$ is unital associative algebra; $V^K$ is a module over $\H(G//K)$. $\H(G) = \cup_K \H(G//K)$. The limit of the units in the small Hecke algebras as $K$ becomes smaller tends to the delta function at the identity, which is not an element of $\H(G)$ unless $G$ is discrete - in which case it serves as a unit $(\H(G) = \H(G//\{e\})$.\\

Is there a relation between this Hecke algebra and the archimedean Hecke algebra of distributions supported at the identity? Relation deeper than ``[adjectives] Representations of group = [adjectives] Representations of modules''?\\

Maybe further, relation between Hecke operators on modular forms and the local ones? How do you understand Hecke operator on modular forms in this way? Can you see local Hecke operator as some kind of averaging?\\

Want to also discuss some unrelated Galois cohomology stuff.



\end{document}
