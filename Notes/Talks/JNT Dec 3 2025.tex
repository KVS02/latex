\documentclass{amsart}

\usepackage{amsmath,amssymb,amsfonts,amscd}
\usepackage[all,cmtip]{xy}
\usepackage{enumerate}
\usepackage{ amssymb, latexsym, amsmath}
%\setcounter{MaxMatrixCols}{30}%
\usepackage{fullpage}
\usepackage{url}
\usepackage{hyperref}
\usepackage{ marvosym }
\usepackage{quiver}
~


\numberwithin{equation}{section}

%\usepackage{cjw-latex}

%\providecommand{\U}[1]{\protect\rule{.1in}{.1in}}

\theoremstyle{plain}
\newtheorem{theorem}{Theorem}[section]
\newtheorem*{thm}{Theorem}
\newtheorem{proposition}[theorem]{Proposition}
\newtheorem{lemma}[theorem]{Lemma}
\newtheorem{conjecture}[theorem]{Conjecture}
\newtheorem{corollary}[theorem]{Corollary}
\newtheorem{algorithm}[theorem]{Algorithm}
\newtheorem{axiom}[theorem]{Axiom}
\newtheorem{criterion}[theorem]{Criterion}
%\newtheorem{conjecture}[theorem]{Conjecture}
\newtheorem{wildconjecture}[theorem]{Wild Conjecture}


\theoremstyle{definition}
\newtheorem{definition}[theorem]{Definition}
\newtheorem*{dfn}{Definition}
\newtheorem{condition}[theorem]{Condition}
\newtheorem{example}[theorem]{Example}
\newtheorem{exercise}[theorem]{Exercise}
\newtheorem{notation}[theorem]{Notation}
\newtheorem{question}[theorem]{Question}
\newtheorem{problem}[theorem]{Problem}
\newtheorem{solution}[theorem]{Solution}




\theoremstyle{remark}
\newtheorem{remark}[theorem]{Remark}
\newtheorem{remarks}[theorem]{Remarks}
\newtheorem{summary}[theorem]{Summary}
\newtheorem{observation}[theorem]{Observation}
\newtheorem{conclusion}[theorem]{Conclusion}
\newtheorem{acknowledgement}[theorem]{Acknowledgement}
\newtheorem{case}[theorem]{Case}
\newtheorem{claim}[theorem]{Claim}



\makeatletter
\newcommand{\rmnum}[1]{\romannumeral #1}
\newcommand{\Rmnum}[1]{\expandafter\@slowromancap\romannumeral #1@}
\makeatother
\newcommand{\Aut}{\operatorname{Aut}}
\newcommand{\Ext}{\operatorname{Ext}}
\newcommand{\Tor}{\operatorname{Tor}}
\newcommand{\Z}{\mathbb{Z}}
\newcommand{\osum}{\oplus}
\newcommand{\aff}{\operatorname{Aff}}

\newcommand{\Id}{\operatorname{Id}}
\newcommand{\concat}{(\Id \to \Omega \Sigma)}
\newcommand{\eval}{(\Sigma\Omega \to \Id)}
\newcommand{\Q}{\mathbb{Q}}
\newcommand{\F}{\mathbb{F}}
\newcommand{\N}{\mathbb{N}}
\newcommand{\g}{\mathfrak{g}}
\newcommand{\n}{\mathfrak{n}}
\newcommand{\h}{\mathfrak{h}}
\newcommand{\p}{\mathfrak{p}}
\newcommand{\q}{\mathfrak{q}}
\newcommand{\m}{\mathfrak{m}}
\newcommand{\e}{\mathfrak{e}}
\newcommand{\f}{\mathfrak{f}}
\newcommand{\V}{\mathscr{V}}
\newcommand{\D}{\mathscr{D}}
\renewcommand{\S}{\mathscr{S}}
\newcommand{\Sch}{\mathscr{C}}
\newcommand{\Y}{\mathscr{Y}}
\renewcommand{\u}{\mathfrak{u}}
\newcommand{\set}[2]{ \left\{ {#1} \, \left| \, {#2} \right\}\right.}
\newcommand{\A}{\mathbb{A}}
\newcommand{\X}{\mathcal{X}}
\renewcommand{\Im}{\operatorname{Im}}
\newcommand{\T}{\mathscr{T}}
\newcommand{\HH}{\mathscr{H}}
\newcommand{\PGL}{\operatorname{PGL}}
\newcommand{\righthookarrow}{\hookrightarrow}
\newcommand{\sign}{\operatorname{sign}}
\renewcommand{\_}[2]{\underbrace{#1}_{#2}}
\renewcommand{\^}[2]{\overbrace{#1}_{#2}}
\newcommand{\curverightarrow}{\curvearrowright}

\newcommand{\z}{\mathfrak{z}}
\renewcommand{\sl}{\mathfrak{sl}}
\newcommand{\R}{\mathbb{R}}
\newcommand{\RP}{\mathbb{RP}}
\renewcommand{\P}{\mathbb{P}}
\newcommand{\C}{\mathbb{C}}
\renewcommand{\H}{\mathcal{H}}
\newcommand{\Ind}{\operatorname{Ind}}
\newcommand{\Tr}{\operatorname{Tr}}
\newcommand{\Orb}{\mathcal{O}}
\renewcommand{\k}{\mbox{\Fontauri k}}
\newcommand{\Ad}{\operatorname{Ad}}
\newcommand{\ad}{\operatorname{ad}}
\newcommand{\Hom}{\operatorname{Hom}}
\newcommand{\Ker}{\operatorname{Ker}}
\newcommand{\End}{\operatorname{End}}
\newcommand{\supp}{\operatorname{supp}}
\newcommand{\Stab}{\mbox{Stab}}
\newcommand{\Lie}{\operatorname{Lie}}
\newcommand{\GL}{\operatorname{GL}}
\newcommand{\SL}{\operatorname{SL}}
\newcommand{\Sp}{\operatorname{Sp}}
\newcommand{\Mp}{\operatorname{Mp}}
\renewcommand{\a}{\mathfrak{a}}
\newcommand{\Wh}{\operatorname{Wh}}
\newcommand{\Span}{\operatorname{Span}}
\newcommand{\WF}{\operatorname{WF}}
\newcommand{\AC}{\operatorname{AC}}
\newcommand{\Lin}{\operatorname{Lin}}
\newcommand{\Diff}{\operatorname{Diff}}
\newcommand{\Gen}{\operatorname{Gen\, Hom}}
\newcommand{\Her}{\operatorname{Her}}
\newcommand{\pivot}{&}
\newcommand{\sgn}{\operatorname{sgn}}
%\newcommand{C^{\infty}}{DS}
\newcommand{\vsp}{{\vspace{0.2in}}}



\title{Notes}

\author{Sudharshan K V}
\begin{document}
\maketitle

\section*{Some general results}

\subsection*{Structure theory}

$M$ is a finitely generated torsion $\Lambda$-module.

1. The $\Lambda$-rank of $M$ is well defined $\Rightarrow$ there is an injection with torsion cokernel.

2. $M_\Gamma$ is finite if and only if $M^\Gamma$ is finite for $M$ torsion.

\subsection*{Galois groups}

Let $F_\infty/F$ be a $\mathbf{Z}_p$-extension, $F_n = F_\infty^{\Gamma_n}$.  
Let $M_\infty/F$ be a Galois extension such that $M_\infty/F_\infty$ is pro-$p$, abelian.  
Set $X := \operatorname{Gal}(M_\infty/F_\infty)$. Then $X$ is a $\Lambda$-module and let $M_n$ be the maximal abelian extension of $F_n$ inside $M_\infty$. Then
$$
\omega_n X = \operatorname{Gal}(M_\infty/M_n).
$$

\subsection*{Cohomology}

$\Gamma$ is procyclic, so its cohomological dimension is $1$.

For a short exact sequence
$$
0 \to A \to B \to C \to 0
$$
one has
$$
H^0(\Gamma,M) = M^\Gamma,\quad H^1(\Gamma,M) = M_\Gamma.
$$

Then, for an exact sequence $0 \to A \to B \to C \to 0$ (alternatively, use the snake lemma on a diagram with vertical $\Gamma$-coinvariants) one obtains the long exact sequence in cohomology.

\section*{Cyclotomic units}

\textbf{Theorem.} Let $M_n$ be the maximal abelian $p$-extension of $F_n$ unramified outside $p$. Then $M_n/F_\infty$ is finite.

\textit{Proof (sketch).} One has
$$
U_n/E_n \cong \operatorname{Gal}(M_n/L_n)
$$
and $L_n/F_n$ is finite. The $p$-rank of the class group is given by
$$
\text{$p$-rank} = [F_n:\mathbf{Q}]
$$
(Leopoldt conjecture input).

Define $\mathcal{N}_\infty(U_n)$, $\mathcal{N}_\infty(K_n)$ as projective limits under norm maps.

\textbf{Theorem.} One wants
$$
\mathbf{Z}_p\text{-rank } E_n' = [F_n:\mathbf{Q}] - 1.
$$

Let
$$
\alpha_{n,u}: \Gamma_n \to \mathcal{N}_\infty(U_n),\qquad
\alpha_{n,c}: \Gamma_n \to C_\infty
$$
be the natural maps. Then:
\begin{enumerate}
\item $(U_\infty/C_\infty)^{\Gamma_n} = 0$.
\item $\operatorname{Val}(C_\infty) = \mathcal{N}_\infty(U_n)/C_\infty$.
\end{enumerate}
These induce a homomorphism
$$
\Gamma_n \to \mathcal{N}_\infty(U_n)/C_\infty,
$$
and one has
$$
\ker(\alpha_{n,u}) \cong \ker(\alpha_{n,c}) \cong \mathbf{Z}_p
$$
with a trivial Galois action.

\section*{Global units}

\textbf{Theorem.} $L'(E_\infty)$ is principal in $\Lambda(\Gamma)$.

\textbf{Proposition (Iwasawa).} $X_{\Gamma_0}$ has no non-zero finite $\Lambda(\Gamma_0)$-submodule, where $\Gamma_0 = \operatorname{Gal}(F_\infty/F_0)$.

Use this without proof.

There is a map $E_\infty \to \Lambda(G)$ with torsion cokernel, so $\operatorname{rk}_\Lambda(E_\infty)$ is well defined and equal to $1$. Thus,
$$
0 \to E_\infty \to \Lambda(G) \to Q \to 0
$$
for some finite $\Lambda(G)$-module $Q$. One wants $Q = 0$. If $Q_{\Gamma_0} = 0$ then $Q = 0$ (Nakayama lemma).

From
$$
0 \to E_\infty \to \Lambda(G) \to Q \to 0
$$
taking $\Gamma_0$-coinvariants gives an exact sequence of $\mathbf{Z}_p$-modules
$$
0 \to (E_\infty)_{\Gamma_0} \to \Lambda(G)_{\Gamma_0} \to Q_{\Gamma_0} \to 0,
$$
and $\Lambda(G)_{\Gamma_0}$ is free, so $Q_{\Gamma_0}$ is the $p$-torsion submodule of $(E_\infty)_{\Gamma_0}$. Now one shows $(E_\infty)_{\Gamma_0}$ has no torsion.

There is an exact sequence
$$
1 \to V_\infty \to U_\infty \to \operatorname{Gal}(M_\infty/L_\infty) \to 0.
$$
Then $\operatorname{Gal}(M_\infty/L_\infty)_{\Gamma_0} = 0$, and $\operatorname{Gal}(M_\infty/L_\infty)^{\Gamma_0} \le X_{\Gamma_0}$. Using class field theory and Leopoldt, one deduces $X_{\Gamma_0}$ is finite, hence $\operatorname{Gal}(M_\infty/L_\infty)_{\Gamma_0} = 0$.

From various norm and valuation maps one gets an exact sequence
$$
0 \to \mathcal{N}_\infty(E_n) \to \mathcal{N}_\infty(U_n) \to \mathcal{N}_\infty(U_n')/\mathcal{N}_\infty(E_n) \to 0,
$$
so
$$
\operatorname{Gal}(M_\infty/L_\infty)_{\Gamma_n} \cong \mathcal{N}_\infty(U_n)/\mathcal{N}_\infty(E_n).
$$
Thus there is an exact sequence
$$
0 \to \operatorname{Gal}(M_\infty/L_\infty) \to X_\infty \to Y_\infty \to 0,
$$
with
$$
(X_\infty)_{\Gamma_n} = \operatorname{Gal}(M_n/F_\infty),\quad
(Y_\infty)_{\Gamma_n} = \operatorname{Gal}(L_n/F_n)\times\operatorname{Gal}(L_nF_\infty/F_\infty).
$$
By class field theory
$$
\operatorname{Gal}(M_n/L_nF_\infty) \cong \mathcal{N}_\infty(U_n)/E_n,
$$
and
$$
Y_\infty \cong E_n / \mathcal{N}_\infty(E_n)
$$
for large $n$, where $Y_\infty$ is the largest finite $\Lambda(\Gamma_0)$-submodule of $X_\infty$.

\section*{Iwasawa $L$-functions}

One has
$$
L'(E_\infty) = \alpha \Lambda(G),\qquad
L'(C_\infty) = \beta \Lambda(G),
$$
with $\alpha\beta = \zeta_p'(0,1)$ for some $\beta$. There is a canonical map
$$
T: E_\infty/C_\infty = \Lambda(G)/\beta\Lambda(G)
$$
inducing an isomorphism
$$
\mathcal{N}_\infty(E_n)/C_\infty \cong R_n/pR_n.
$$

If (First Theorem) $K_\infty/K_0$ is the maximal abelian $p$-extension, and $M_\infty$ is the maximal abelian $p$-extension of $K_n$, then
$$
\operatorname{Gal}(M_\infty/K_\infty)_{\Gamma_n} \cong \operatorname{Gal}(M_n/K_\infty),
$$
and $\widehat{K}_\infty$ denotes the $p$-adic completion. Let
$$
\mathcal{Z}_\infty = \varprojlim K_n
$$
with inverse limit taken with respect to the norm maps. Then
$$
\mathcal{Z}_\infty \cong \operatorname{Gal}(M_\infty/K_\infty),
$$
and
$$
\mathcal{N}_\infty(K_n) \cong \operatorname{Gal}(M_n/K_\infty).
$$
From local class field theory one gets a natural map
$$
\mathcal{Z}_\infty \to \mathcal{N}_\infty(K_n)
$$
whose image identifies with $\mathcal{Z}_\infty = \mathcal{N}_\infty(K).$

On the global side, $W_n$ is the subgroup of $U_n$ generated by cyclotomic units and their Galois conjugates, with $W_0 = \mathcal{N}_\infty(K_n)$, and valuation induces
$$
0 \to \dots \to W_n \to \mathbf{Z}_p \to 0.
$$

Cyclotomic units: taking $\zeta_{p^n}$ and its conjugates, one gets
$$
0 \to D_n \to W_n \to \mathbf{Z}_p \to 0.
$$
Since $(D_n : D_1)$ is prime to $p$, the projective limit $D_\infty\otimes \mathbf{Z}_p$ is isomorphic to its closure $C_\infty'$ inside $U_\infty$; the $\mathbf{Z}_p$-rank of $E_n$ equals the $\mathbf{Z}$-rank of $D_n$ by Leopoldt.

Finally, snake lemma arguments on the diagram relating $U_n$, $K_n^\times$, and their completions show
$$
(U_\infty/C_\infty)_{\Gamma_n}
$$
is finite but an elementary $\Lambda(G)$-module has no non-zero finite submodule, so $U_\infty/C_\infty = 0$. This implies $\ker(\alpha_{n,u}) \cong \ker(\alpha_{n,c})$, and comparing $\mathbf{Z}_p$-ranks in the exact sequences shows the desired injectivity.



\end{document}
